\begin{acknowledgements}
Το παρόν έργο διεξήχθη στο Εργαστήριο Ευφυών Συστημάτων, Περιεχομένου και Αλληλεπίδρασης του τμήματος Ηλεκτρολόγων Μηχανικών και Μηχανικών Η/Υ του Εθνικού Μετσόβιου Πολυτεχνείου, υπό την επίβλεψη και καθοδήγηση του καθηγητή Δρ. Γεωργίου Στάμου. Η διπλωματική αυτή θεμελιώνεται στις κυριότερες αρχές που διέπουν την Ανάπτυξη Διαδικτυακών Εφαρμογών, αλλά ταυτοχρόνως διερευνά την ομορφιά και το μεγαλείο των πιο σύγχρονων διαδικτυακών τεχνολογιών του σήμερα.

Θα ήθελα να ευχαριστήσω τους καθηγητές και επιβλέποντές μου Δρ. Γ. Στάμου και Δρ. Β. Τζουβάρα για την ώθηση που μου έδωσαν ώστε να εμπνευστώ την κεντρική ιδέα του παρόντος έργου και την εμπιστοσύνη που μου έδειξαν στην πρωτοβουλία μου αυτή. Ένα μεγάλο ευχαριστώ αξίζει και σε όλους τους υπόλοιπους καθηγητές μου, για την διδασκαλία τους και τις γνώσεις που μου μετέδωσαν κατά τη διάρκεια των σπουδών μου. 

Ευχαριστώ ιδιαιτέρως την μεταπτυχιακό Μαρία Ράλλη για την στήριξή της από την αρχή αυτής της προσπάθειας, την ενθαρρυντική στάση της, τις συβουλές της και τις γνώσεις της. Η καθοδήγησή της και η καλή διάθεση που μου έδειξε με βοήθησαν σημαντικά στην επίτευξη του έργου μου.

Τέλος, αφιερώνω το παρόν έργο στην οικογένειά μου, για την αμέριστη στήριξη και εμπιστοσύνη που μου έδειξαν στις επιλογές και στις σπουδές μου. Ευχαριστώ τον αδερφό και την αδερφή μου που συνεισέφεραν με το δικό τους τρόπο ο καθένας στην προσπάθειά μου αυτή, καθώς επίσης και τους γονείς μου, οι οποίοι μου έμαθαν να ακολουθώ πάντοτε τα όνειρά μου και να επιτρατεύομαι τη γνώση ως σύμμαχο σε όλα μου τα προβλήματα. Τίποτε δεν είναι πιο πολύτιμο από τη γνώση. Σας είμαι ευγνώμων για τον άνθρωπο στον οποίο έχω εξελιχθεί σήμερα.
\end{acknowledgements}


\begin{abstract}

Τα  προβλήματα  επίτευξης  επικοινωνίας σήμερα έχουν πλέον εκλείψει. Χάρη στην πληθώρα των μέσων κοινωνικής δικτύωσης, μπορεί κανείς να αναζητήσει, να συνομιλήσει ή να γνωρίσει άλλα άτομα άμεσα. Ωστόσο, η απότομη αυτή στροφή στην διαπροσωπική επικοινωνία έχει διεισδύσει υπερβολικά στην προσωπική ζωή δημιουργώντας νέα εμπόδια και κινδύνους που περισσότερο δυσχεραίνουν, παρά καλύπτουν τις ανάγκες των χρηστών για επικοινωνία. Η παρούσα διπλωματική εργασία διερευνά τρόπους αντιμετώπισης του παραπάνω φαινομένου, μέσα από την ανάπτυξη μιας καινοτόμου εφαρμογής που στηρίζεται σε πληθοποριστικές πρακτικές. Η εφαρμογή θα εισάγει τους χρήστες της σε μια νέα εμπειρία, η οποία σκοπό έχει την ανάδειξη των δυνατοτήτων των χρηστών της, καθώς επίσης και την προβολή των σημερινών πολιτισμικών γεγονότων. Αυτό θα είναι εφικτό αξιοποιώντας την γνώση και την εμπειρία των χρηστών εντός της εφαρμογής για την άσκηση κριτικής, τον σχολιασμό και την συζήτηση με επίκεντρο εκδηλώσεις που αφορούν την πολιτισμική κοινότητα. Με αυτό τον τρόπο, αντιστρέφονται οι ρόλοι, καθώς η εφαρμογή παύει να αποτελεί τον πομπό όπως συνηθίζεται σήμερα, αλλά γίνεται ο δέκτης της πληροφορίας. Οι χρήστες με τη σειρά τους, θα αποτελούν την πηγή της πληροφορίας, χάρη στη γνώση που θα παρέχουν μέσα από τη χρήση της εφαρμογής. Αυτό θα οδηγήσει στη διαμόρφωση μιας πιο ενεργητικής στάσης απέναντι στα πολιτισμικά και κοινωνικά δρώμενα, με αποτέλεσμα την συμμετοχή του κοινωνικού συνόλου σε περισσότερα γεγονότα και εκδηλώσεις.
\newline
\indent
Η παραπληροφόρηση και η μονόπλευρη διαφήμιση στις μέρες μας, αφήνει ελάχιστα περιθώρια στους νέους που θέλουν να εξερευνήσουν την πόλη τους, να δοκιμάσουν νέες εμπειρίες και να ξεφύγουν από τον τετριμμένο τρόπο διασκέδασης. Ακόμη, η επιλογή ενός μέρους με βάση τις προτιμήσεις, μειώνει περισσότερο τις επιλογές καθώς οι ενδιαφερόμενοι επιλέγουν περισσότερο με βάση τι είναι σίγουρο και γνωστό, και λιγότερο με βάση τις προσωπικές τους επιθυμίες. Στόχος μας είναι η ανάπτυξη ενός καινοτόμου τρόπου γεφύρωσης του χάσματος που δημιουργούν οι σημερινοί ψηφιακοί κίνδυνοι, μέσα από τη χρήση σύγχρονων τεχνολογιών και πρωτοπόρων ιδεών. Η προσοχή μας θα επικεντρωθεί στη διαμόρφωση μιας εφαρμογής, της οποίας πρωταρχικός σκοπός είναι η διευκόλυνση της διαδικασίας της επικοινωνίας και της διασκέδασης με τη χρήση ενός νέου συστήματος μηνυμάτων σε τοποθεσίες επάνω σε χάρτη. Με αυτό τον τρόπο οι χρήστες θα μπορούν να πληροφορηθούν άμεσα για το τι συμβαίνει στην περιοχή ή το μέρος ενδιαφέροντος, να αλληλεπιδράσουν σε πραγματικό χρόνο με τους ευρισκόμενους την παρούσα χρονική στιγμή εκεί, να εξερευνήσουν νέα μέρη και να απολαύσουν τη διαδικασία της οργάνωσης της διασκέδασης. Πρωτίστως όμως, στόχος της εφαρμογής είναι να άρει τα σημερινά στερεότυπα για τις κοινωνικές ομάδες μεταξύ των νέων, φέρνοντας πιο κοντά το χρήστη με τα ενδιαφέροντά του. Να μετατρέψει σε πλειονότητες τις μειονότητες εκείνων που έχουν παραγκωνιστεί από το υπόλοιπο σύνολο λόγω των διαφορετικών αντιλήψεων και ενδιαφερόντων. Να ρίξει φως σε εκείνα τα μέρη που δεν παίρνουν την αντικειμενική αξία που τους αναλογεί. Να δώσει μάτια σε εκείνα τα άτομα  που επιλέγουν να ακολουθούν τυφλά την υπόλοιπη μάζα επειδή νομίζουν πως είναι μόνοι. Κάθε άτομο, οποιασδήποτε ομάδας, ηλικίας, γένους ή καταγωγής έχει δικαίωμα στον πολιτισμό και στην ενημέρωση. Δουλειά μας είναι να το κάνουμε ευρέως γνωστό.

\begin{keywords}
  Ανάπτυξη Εφαρμογών, Κινητή Εφαρμογή, Εξυπηρετητής, Βάση Δεδομένων, Πληθοπορισμός, \tl{React, React Native, JavaScript, Google Maps, Foursquare, RESTful APIs, Expo, Redux, MERN, Full Stack. }
\end{keywords}

\end{abstract}



\begin{abstracteng}
\tl{The problems of reaching communication today are gone. Thanks to the abundance of social media, one can search, chat or meet other people directly. However, this steep shift in interpersonal communication has penetrated too much into personal life, creating new barriers and dangers that make it harder to address users' needs for communication. This diploma thesis explores ways to address this phenomenon, through the development of an innovative application based on crowdsourcing techniques. The application will introduce its users to a new experience, which aims at highlighting the capabilities of its users, as well as the cultural events that take place today. This will be feasible by leveraging the users' knowledge and experience within the app for reviewing, commenting and discussing community-centered cultural events. In this way, the roles are reversed as the application ceases to be the transmitter of information as it is used to be today, but becomes the receiver of the information. Users, on the other hand, will be the source of this information, thanks to the knowledge they will provide through the application's interface. This will lead to a more active attitude towards cultural and social events.}

\tl{Disinformation and one-sided advertising nowadays leaves little room for young people to explore their city, try new experiences and escape the trivial way of having fun. In addition, choosing a preference-based part reduces ones choices, as people choose more on the basis of what is certain and familiar, and less on the basis of their personal wishes. Our goal is to present an innovative way to bridge the gap created by today's digital dangers through the use of modern technologies and pioneering ideas. Our focus will be the developing an application, whose primary purpose is to facilitate the communication and entertainment process by using a new messaging system on map locations. In this way, users can instantly learn about what's happening in their area or at the place of interest, interact in real time with those currently present at the event, explore new places, and enjoy the process of organizing their plans for having fun. Above all, however, the main goal of this application is to lift the current stereotypes for social groups among young people, bringing the user closer to their interests. Convert to minorities those who have been crowded out of the rest due to differences based on their interests. To shed light on those places that do not get the value they deserve. Give eyes to those people who choose to blindly follow the rest of the mass, because they think they are alone. Every person, of any age, gender or background, has a right into culture and information. Our job is to make it widely known.}



\begin{keywordseng}
    \tl{App Development, Mobile Application, Server, Database, Crowdsourcing, React, React Native, JavaScript, Node, Express, Mongo, Google Maps, Foursquare, RESTful APIs, Expo, Redux, MERN, Full Stack. }
\end{keywordseng}

\end{abstracteng}
