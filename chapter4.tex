<<<<<<< HEAD
\chapter{Σχεδίαση Συστήματος}
\label{chap4}
=======
\chapter{< τίτλος που αφορά την επίλυση του προβλήματος >, π.χ.: Επεξεργασία πιθανοτικών ερωτημάτων \tl{\textit{k}}-εγγύτερων γειτόνων}
\label{chap5}
>>>>>>> acacc83a12cc6f1be99d6d3fb0df8b0ed3fa708b

\section{Εισαγωγή}

Εδώ εξηγούμε ότι θα περιγράψουμε το κύριο κομμάτι της διπλωματικής μας, που είναι στην ουσία η ανάπτυξη μεθόδων και αλγορίθμων για την επίλυση του προβλήματος που ορίσαμε στο προηγούμενο κεφάλαιο. 

\section{<τίτλος που αφορά τεχνική/αλγόριθμο 1, π.χ.: Εφαρμογή φίλτρων βάσει καννάβου>}

Περιγραφές μπορούν να γίνουν βάζοντας ενδεικτικά κομμάτια κώδικα ή ψευδοκώδικα, όπως ο Αλγόριθμος \ref{alg:processing} και περιγράφοντάς τα με λόγια. Μην ξεχνάτε να δίνετε πάντα παραδείγματα για το πώς τρέχει ένα κομμάτι ψευδοκώδικα π.χ. για έναν αλγόριθμο.
Αν έχετε επίσης θεωρήματα που αποδεικνύουν κάποια αποτελέσματα των τεχνικών/αλγορίθμων, τα βάζετε εδώ. 

\begin{otherlanguage}{english}
\begin{algorithm}
\caption{\ \ \ Probabilistic $k\theta NN$ Monitoring}
\begin{algorithmic}[1]
\begin{footnotesize}

\STATE{\bf Procedure} {\em VerifyCandidate} (focal query point $q$, threshold $\theta$, object $o$, list of auxiliary objects $P$, distance $kMAXDIST$) 

\IF { $\Phi(o, kMAXDIST) \geq \theta$ {\bf and} $L_2(q, o) \leq L_2(q, P.$top()) }

\STATE {$P$.pop()};   \ \ \ \ \ \ \ \ {\em //Replace the most extreme element in $P$, since candidate $o$ ... }

\STATE {$P$.push($o$)};  \ \ \ \ \ {\em //... has enough probability and has its mean closer to focal $q$ }

\ENDIF

\STATE {\bf End Procedure}


\end{footnotesize}
\end{algorithmic}
\label{alg:processing}
\end{algorithm}
\end{otherlanguage}


\section{<τίτλος που αφορά τεχνική/αλγόριθμο 2, π.χ.: Βελτιστοποίηση αναζητήσεων>}

...


