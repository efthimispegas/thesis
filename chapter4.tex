\chapter{Σχεδίαση Συστήματος}
\label{chap4}

Στο προηγούμενο κεφάλαιο έγινε μια εισαγωγική αναφορά στα δομικά μέρη του συστήματος της εφαρμογής, δηλάδή του \tl{frontend} (\tl{client}) και του \tl{backend} (\tl{server} και βάση δεδομένων). Αναπτύχθηκαν οι λειτουργικότητες της εφαρμογής και οι αντίστοιχες οθόνες που θα τις καλύπτουν. Επίσης, έγινε μια αναφορά στα συστατικά του \tl{server} και πως αυτά αλληλεπιδρούν με τη βάση δεδομένων για την λήψη και αποστολή δεδομένων από και προς τον \tl{client}.

Αυτό το κεφάλαιο θα επικεντρωθεί στην σχεδίαση των υποδομών της εφαρμογής. Συγκεκριμένα, θα παρουσιαστούν τα διαγράμματα ροής για τα σενάρια χρήσης της εφαρμογής και θα αναπτυχθούν τα επίμαχα σημεία που χρήζουν της περισσσότερης προσοχής. Έπειτα, θα παρουσιαστούν τα σχέδια των διεπιφανειών που θα συνιστούν κάθε οθόνη της εφαρμογής (βλ. ενότητα 4.1). Στην ενότητα 4.2, θα γίνει αναφορά στην πλευρά του \tl{server}. Θα παρουσιαστεί η διαδικασία σχεδίασης των εξυπηρετητών αιτημάτων και θα μελετηθούν οι τακτικές και οι τεχνολογίες που χρησιμοποιήθηκαν. Τέλος, στην ενότητα 4.3, θα γίνει μια εκτενής ανάλυση του τρόπου με τον οποίο σχεδιάστηκε η βάση δεδομένων για να ανταποκρίνεται βέλτιστα στις απαιτήσεις της εφαρμογής.


\section{Σχεδίαση Μοντέλων των Διεπιφανειών της Εφαρμογής}




\section{Σχεδίαση Συστήματος Εξυπηρέτησης Αιτημάτων}



\section{Σχεδίαση Μοντέλου Βάσης Δεδομένων}



