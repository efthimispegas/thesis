\chapter{Επίλογος}
\label{chap7}

% Εδώ εξηγούμε ότι θα συνοψίσουμε την μελέτη που εκπονήθηκε στα πλαίσια της διπλωματικής.
Στο παρόν κεφάλαιο συνοψίζεται η συνολική προσπάθεια για την σχεδίαση και την υλοποίηση του συστήματος της εφαρμογής μηνυμάτων σε διεπαφή χάρτη με τη βοήθεια του πληθοπορισμού. Θα εκφραστούν οι σκέψεις για μελλοντικές επεκτάσεις της εφαρμογής, για περαιτέρω λειτουργικότητες που μπορούν να προστεθούν, καθώς επίσης και για ανάγκες που μπορούν να εξυπηρετηθούν.

\section{Τελικό Σύστημα}
Το σύστημα σχεδιάστηκε με τέτοιο τρόπο ώστε να είναι ευέλικτο και γρήγορο στη χρήση. Από την πρώτη επαφή με την εφαρμογή, ο χρήστης εισάγεται στο νόημα της εφαρμογής άμεσα και αναλαμβάνει το ρόλο που του αναθέτει η εφαρμογή αβίαστα και χωρίς δυσκολία. Σε αυτό συνέβαλλαν τόσο οι σχεδιαστικές τεχνικές που χρησιμοποιήθηκαν, όσο και οι σύγχρονες τεχνολογίες με τις οποίες υλοποιήθηκαν οι διάφορες διεπιφάνειες της εφαρμογής. Η εξέλιξη πλέον των αρχιτεκτονικών σχεδίασης, του διαδικτύου, αλλά και των δυνατοτήτων των καθημερινών υπολογιστών είναι τέτοιες που η ανάπτυξη εφαρμογών δεν απαιτεί πολλά έξοδα, ενώ οι εφαρμογές δεν στερούνται τίποτα σε ταχύτητα ή πολυπλοκότητα. Επιπρόσθετα, η εκμάθηση των \tl{frameworks} είναι πλέον σχετικά σύντομη και εύκολη, ενώ οι δυνατότητες που παρέχουν είναι πολύ εξελιγμένες και έτσι πολλές εφαρμογές μπορούν να παραχθούν σχετικά εύκολα και αποδοτικά. Μέσω αυτών των δυνατοτήτων το υλοποιημένο σύστημα μπόρεσε να παράσχει στους χρήστες της εφαρμογής τη δυνατότητα να δημιουργούν δημοσιεύσεις και σχόλια, ενώ δόθηκε και ένα ολοκληρωμένο κοινωνικό δίκτυο με σκοπό την επικοινωνία και παρακολούθηση των δράσεων μεταξύ χρηστών και εφαρμογής. Η δωρεάν διάθεση της εφαρμογής σε όλους τους χρήστες, καθιστά προσβάσιμη την απόκτησή της και τη χρήση της σε οποιαδήποτε τοποθεσία διαθέτει κάποιο τρόπο σύνδεσης στο διαδίκτυο. Η απλή επίσης σχεδιαστική τεχνοτροπία βοηθάει τους άπειρους χρήστες να μάθουν γρήγορα το σύστημα, χωρίς έτσι να απαιτείται μεγάλος χρόνος εκμάθησης.


\section{Μελλοντικές επεκτάσεις}

Το παραπάνω σύστημα μπορεί να χρησιμοποιηθεί ως μια βάση ώστε να
εμπλουτιστεί με νέες δυνατότητες οι οποίες να παρέχουν στο χρήστη μια πιο ολοκληρωμένη εμπειρία σχετικά με την ενημέρωση σε θέματα πολιτισμικού περιεχομένου. Μερικά μελλοντικά θέματα προς διερεύνηση θα μπορούσαν να είναι:

\begin{itemize}
    \item Δυνατότητα επισύναψης \tl{3D} αντικειμένων στις δημοσιεύσεις του χρήστη, τα οποία θα μπορούν να είναι φωτογραφίες,αρχεία βίντεο ή άλλης υποστηριζόμενης μορφής αρχεία (πχ. \tl{.fbx, .obj, .3ds} κλπ.).
    \item Δυνατότητα αλληλεπίδρασης του χρήστη με τα \tl{3D} αρχεία μιας δημοσίευσης, όπως προβολή, σχολιασμός και απάντηση με νέο μήνυμα αρχείου.
    \item Επιλογή αλληλεπίδρασης με την προσθήκη ενός \tl{like button}.
    \item Εμφάνιση χρονικής διάρκειας ισχύος στις λεπτομέρειες κάθε \tl{hotspot} και διαγραφή μετά τη λήξη αυτού, ο χάρτης ανανεώνεται διαρκώς με τα ισχύοντα \tl{hotspots}.
    \item Προσθήκη χρηστών σε λίστα φίλων για γρήγορη εύρεση και προβολή των δημοσιεύσεών τους.
    \item Δυνατότητα αποστολής μυνημάτων σε φίλους με την βοήθεια ενσωματωμένης πλατφόρμας γι' αυτό το σκοπό.
\end{itemize}
