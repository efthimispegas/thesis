\documentclass[11pt,a4paper,english,greek,twoside]{dblab-thesis}
\usepackage{epsfig}
%\usepackage[english,greek]{babel}
%\usepackage[T1]{fontenc}
%\usepackage[iso-8859-7]{inputenc}
%\usepackage{graphicx}
%\DeclareGraphicsRule{.tif}{bmp}{}{}
\usepackage{titlesec}
<<<<<<< HEAD
\setcounter{secnumdepth}{3}
\setcounter{tocdepth}{3}
=======
\setcounter{secnumdepth}{4}
>>>>>>> acacc83a12cc6f1be99d6d3fb0df8b0ed3fa708b
\usepackage{enumitem}
\usepackage{xcolor}
\usepackage{listings}
\usepackage{color}
<<<<<<< HEAD
\usepackage{courier}
=======
>>>>>>> acacc83a12cc6f1be99d6d3fb0df8b0ed3fa708b

\definecolor{codegreen}{rgb}{0.15,0.6,0.14}
\definecolor{codegray}{rgb}{0.5,0.5,0.5}
\definecolor{codepurple}{rgb}{0.58,0,0.82}
\definecolor{codeorange}{rgb}{0.97,0.39,.05}
\definecolor{codelightyellow}{rgb}{0.88,0.95,0.56}
\definecolor{codelightgrey}{rgb}{0.27, 0.27,0.27}
\definecolor{codebluegreen}{rgb}{0.24,0.59,0.35}
\definecolor{codeblue}{rgb}{0,0,0.6}
\definecolor{codered}{rgb}{0.6,0,0}
\definecolor{backcolor}{rgb}{0.95,0.95,0.92}
\definecolor{darkgray}{rgb}{.4,.4,.4}
\definecolor{purple}{rgb}{0.65, 0.12, 0.82}

\lstdefinestyle{mystyle}{
    backgroundcolor=\color{backcolor},  
    commentstyle={\small\itshape},
    commentstyle=\color{codered}, %comments
    keywordstyle = [1]\color{codeblue}, %reserved words 1
    keywordstyle = [2]\color{codepurple}, %reserved words 2
    keywordstyle = [3]\color{codebluegreen}, %classes
    keywordstyle = [4]\color{codelightyellow}, %functions
    keywordstyle = [5]\color{codelightgrey}, %props
    numberstyle=\tiny\color{codegray},
    stringstyle=\color{codeorange},
<<<<<<< HEAD
    basicstyle=\footnotesize\ttfamily,
=======
    basicstyle=\footnotesize,
>>>>>>> acacc83a12cc6f1be99d6d3fb0df8b0ed3fa708b
    breakatwhitespace=false,         
    breaklines=true,                 
    captionpos=b,                    
    keepspaces=true,                 
    numbers=left,                    
    numbersep=5pt,                  
    showspaces=false,                
    showstringspaces=false,
    showtabs=false,                  
    tabsize=2
}
<<<<<<< HEAD

\lstdefinelanguage{HTML5}{
  morekeywords=[1]{title, html, head, body, button, script, var, function},
  morekeywords=[5]{main, put, printLn, fibonacci, get, containsKey},
  identifierstyle=\color{black},
  sensitive=true,
  morecomment = [l]{//},
  morecomment = [s]{/*}{*/},
  morecomment = [s]{/**}{*/},
  commentstyle=\color{codegreen}\ttfamily,
  stringstyle=\color{codeorange}\ttfamily,
  morestring=[b]',
  morestring=[b]"
}

\lstdefinelanguage{JavaScript}{
  morekeywords=[1]{function, null, false, true, new},
  morekeywords=[2]{return, if},
  morekeywords=[3]{LocalStrategy},
  identifierstyle=\color{black},
  sensitive=true,
  morecomment = [l]{//},
  morecomment = [s]{/*}{*/},
  morecomment = [s]{/**}{*/},
  commentstyle=\color{codegreen}\ttfamily,
  stringstyle=\color{codeorange}\ttfamily,
  morestring=[b]',
  morestring=[b]"
}

\lstdefinelanguage{React}{
  morekeywords=[1]{class, extends, constructor, super, this, const, var, let, div, p, button},
  morekeywords=[2]{return},
  morekeywords=[3]{Board, React, Component, Square, Array},
  morekeywords=[5]{render, renderSquare, handleClick, setState, slice, fill},
  morekeywords=[5]{return},
  identifierstyle=\color{black},
  sensitive=true,
  morecomment = [l]{//},
  morecomment = [s]{/*}{*/},
  morecomment = [s]{/**}{*/},
  commentstyle=\color{codegreen}\ttfamily,
  stringstyle=\color{codeorange}\ttfamily,
  morestring=[b]',
  morestring=[b]"
}
=======
>>>>>>> acacc83a12cc6f1be99d6d3fb0df8b0ed3fa708b
 
\lstdefinelanguage{Swift}{
  morekeywords=[1]{class, export, boolean, throw, implements, import, this,@IBOutlet,@IBAction, weak, override func, super,var, let, override, weak, func, false, true},
  morekeywords=[2]{import, typeof, try, new, catch, function, return, null, catch, switch, if, in, while, do, else, case, break},
  morekeywords=[3]{UIKit, AVFoundation, ViewController, Any},
  morekeywords=[5]{sender, main, description, isHidden},
  identifierstyle=\color{black},
  sensitive=true,
  morecomment = [l]{//},
  morecomment = [s]{/*}{*/},
  morecomment = [s]{/**}{*/},
  commentstyle=\color{purple}\ttfamily,
  stringstyle=\color{codeorange}\ttfamily,
  morestring=[b]',
  morestring=[b]"
}

\lstdefinelanguage{Java}{
  morekeywords=[1]{class, import, export, boolean, implements, extends, this, func, super, var, let, false, true, public, static, @param, @return},
  morekeywords=[2]{try, catch, return, switch, if, in, while, do, else, case, break, else if, new},
  morekeywords=[3]{FibCalculator, Fibonacci, Calculator, Map, Integer, HashMap, String, void},
  morekeywords=[5]{main, put, printLn, fibonacci, get, containsKey},
  identifierstyle=\color{black},
  sensitive=true,
  morecomment = [l]{//},
  morecomment = [s]{/*}{*/},
  morecomment = [s]{/**}{*/},
  commentstyle=\color{codegreen}\ttfamily,
  stringstyle=\color{codeorange}\ttfamily,
  morestring=[b]',
  morestring=[b]"
}

\lstdefinelanguage{React Native}{
  morekeywords=[1]{class, export, boolean, throw, implements, import, this,@IBOutlet,@IBAction, weak, override func, super,var, let, override, weak, func, false, true},
  morekeywords=[2]{import, typeof, try, new, catch, function, return, null, catch, switch, if, in, while, do, else, case, break},
  morekeywords=[3]{UIKit, AVFoundation, ViewController, Any},
  morekeywords=[5]{sender, main, description, isHidden},
  identifierstyle=\color{black},
  sensitive=false,
  comment=[l]{//},
  morecomment=[s]{/*}{*/},
  commentstyle=\color{purple}\ttfamily,
  stringstyle=\color{codeorange}\ttfamily,
  morestring=[b]',
  morestring=[b]"
}

<<<<<<< HEAD
\lstdefinelanguage{JSON}{
  morekeywords=[1]{true, null},
  identifierstyle=\color{black},
  sensitive=false,
  comment=[l]{//},
  morecomment=[s]{/*}{*/},
  commentstyle=\color{purple}\ttfamily,
  stringstyle=\color{codeorange}\ttfamily,
  morestring=[b]"
}

=======
>>>>>>> acacc83a12cc6f1be99d6d3fb0df8b0ed3fa708b
\lstset{style=mystyle}

\usepackage{minted}
\usepackage{acronym}

\usepackage[explicit]{titlesec}
\usepackage{indentfirst}
\usepackage{verbatim}
\usepackage{amsmath}
\usepackage{subcaption}
\usepackage{amsthm}
\usepackage{amssymb}
\usepackage{epstopdf}
\usepackage{latexsym}
\usepackage{index}
\usepackage{datetime}
\usepackage{textcomp}
\usepackage{graphicx}
\usepackage{url}
\usepackage{array}
\usepackage{algorithm}
\usepackage{algorithmic}
\usepackage{babel}
\usepackage{afterpage}
\usepackage{caption}
%\usepackage{makeidx}
%\bibliographystyle{alpha}
\bibliographystyle{abbrv}

\newindex{default}{idx}{ind}{Ευρετήριο όρων}
\newindex{en}{edx}{end}{Ευρετήριο αγγλικών όρων}
%\makeindex


% Page definitions
%\setlength{\textheight}{23cm} \setlength{\textwidth}{15.5cm}
%\setlength{\oddsidemargin}{0.2cm}
%\setlength{\evensidemargin}{0.2cm} \setlength{\topmargin}{-1.2cm}
%\setlength{\headsep}{1.5cm}

% 1.5 spacing
\renewcommand{\baselinestretch}{1.2}

\newcommand\blankpage{%
    \null
    \thispagestyle{empty}%
    \addtocounter{page}{-1}%
    \newpage}
% latin text (and greek text)
%\newcommand{\prg}[1]{\textlatin{\texttt{#1}}}
\newcommand{\tl}[1]{\textlatin{#1}}
\newcommand{\tg}[1]{\textgreek{#1}}

% typeset short english phrases
\newcommand{\en}[1]{\foreignlanguage{english}{#1}}

% typeset source code
\newcommand{\src}[1]{{\tt\en{#1}}}



% typeset a backslash
\newcommand{\bkslash}{\en{\symbol{92}}}

%typeset infx(a) supx(a) etc
%\newcommand{\infx}[1]{inf_x({#1})}
%\newcommand{\infy}[1]{inf_y({#1})}
%\newcommand{\supx}[1]{sup_x({#1})}
%\newcommand{\supy}[1]{sup_y({#1})}
%\newcommand{\dlt}{\delta}
%\newcommand{\most}{${\cal M}ost$}
%\newcommand{\br}{${\cal B}r$}
\newcommand*\Hide{%
\titleformat{\chapter}[display]
  {}{}{0pt}{\Huge}
\titleformat{\part}
  {}{}{0pt}{}
}
\newtheorem{definition}{Ορισμός}
\newtheorem{proposition}{Πρόταση}
\newtheorem{theorem}{Θεώρημα}
\newtheorem{corollary}{Συμπέρασμα}
\newtheorem{lemma}{Λήμμα}
\newtheorem{example}{Παράδειγμα}
\newtheorem{remark}{Σημείωση}
\newtheorem{notation}{Συμβολισμός}
\newtheorem{law}{Νόμος}
\renewcommand{\thedefinition}{\arabic{chapter}.\arabic{definition}}
\renewcommand{\theproposition}{\arabic{chapter}.\arabic{proposition}}
\renewcommand{\thetheorem}{\arabic{chapter}.\arabic{theorem}}
\renewcommand{\thecorollary}{\arabic{chapter}.\arabic{corollary}}
\renewcommand{\thelemma}{\arabic{chapter}.\arabic{lemma}}
\renewcommand{\theexample}{\arabic{chapter}.\arabic{example}}
\newcommand{\set}[1]{\left\{#1\right\}}
\newcommand{\To}{\Longrightarrow}
\newcommand{\xml}{\en{XML}}


\selectlanguage{greek}
%\selectlanguage{english}
<<<<<<< HEAD
\hyphenation{τμή-μα Επο-μέ-νως χρησι-μο-ποιεί χρη-σιμο-ποιεί-ται}
=======
\hyphenation{τμή-μα Επο-μέ-νως}
>>>>>>> acacc83a12cc6f1be99d6d3fb0df8b0ed3fa708b

\title{Αναζήτηση \textit{\tl{k}}-εγγύτερων γειτόνων μεταξύ περιοχών αβεβαιότητας με κανονική κατανομή}
\author{Χρήστος Κούτρας}
\supervisor{Ιωάννης Βασιλείου}
\TRnumber{ΕΣΒΓΔ-ΔΙΠΛ-2015-03}
\epitropiF{Νεκτάριος Κοζύρης}
\epitropiS{Ιωάννης Θεοδωρίδης}


\begin{document}
\selectlanguage{greek}
\maketitle

\frontmatter
\pagenumbering{roman}
\mainmatter
\begin{acknowledgements}
Το παρόν έργο διεξήχθη στο Εργαστήριο Ευφυών Συστημάτων, Περιεχομένου και Αλληλεπίδρασης του τμήματος Ηλεκτρολόγων Μηχανικών και Μηχανικών Η/Υ του Εθνικού Μετσόβιου Πολυτεχνείου, υπό την επίβλεψη και καθοδήγηση του καθηγητή Δρ. Γεωργίου Στάμου. Η διπλωματική αυτή θεμελιώνεται στις κυριότερες αρχές που διέπουν την Ανάπτυξη Διαδικτυακών Εφαρμογών, αλλά ταυτοχρόνως διερευνά την ομορφιά και το μεγαλείο των πιο σύγχρονων διαδικτυακών τεχνολογιών του σήμερα.

Θα ήθελα να ευχαριστήσω τους καθηγητές και επιβλέποντές μου Δρ. Γ. Στάμου και Δρ. Β. Τζουβάρα για την ώθηση που μου έδωσαν ώστε να εμπνευστώ την κεντρική ιδέα του παρόντος έργου και την εμπιστοσύνη που μου έδειξαν στην πρωτοβουλία μου αυτή. Ένα μεγάλο ευχαριστώ αξίζει και σε όλους τους υπόλοιπους καθηγητές μου, για την διδασκαλία τους και τις γνώσεις που μου μετέδωσαν κατά τη διάρκεια των σπουδών μου. 

Ευχαριστώ ιδιαιτέρως την μεταπτυχιακό Μαρία Ράλλη για την στήριξή της από την αρχή αυτής της προσπάθειας, την ενθαρρυντική στάση της, τις συβουλές της και τις γνώσεις της. Η καθοδήγησή της και η καλή διάθεση που μου έδειξε με βοήθησαν σημαντικά στην επίτευξη του έργου μου.

Τέλος, αφιερώνω το παρόν έργο στην οικογένειά μου, για την αμέριστη στήριξη και εμπιστοσύνη που μου έδειξαν στις επιλογές και στις σπουδές μου. Ευχαριστώ τον αδερφό και την αδερφή μου που συνεισέφεραν με το δικό τους τρόπο ο καθένας στην προσπάθειά μου αυτή, καθώς επίσης και τους γονείς μου, οι οποίοι μου έμαθαν να ακολουθώ πάντοτε τα όνειρά μου και να επιτρατεύομαι τη γνώση ως σύμμαχο σε όλα μου τα προβλήματα. Τίποτε δεν είναι πιο πολύτιμο από τη γνώση. Σας είμαι ευγνώμων για τον άνθρωπο στον οποίο έχω εξελιχθεί σήμερα.
\end{acknowledgements}


\begin{abstract}

Τα  προβλήματα  επίτευξης  επικοινωνίας σήμερα έχουν πλέον εκλείψει. Χάρη στην πληθώρα των μέσων κοινωνικής δικτύωσης, μπορεί κανείς να αναζητήσει, να συνομιλήσει ή να γνωρίσει άλλα άτομα άμεσα. Ωστόσο, η απότομη αυτή στροφή στην διαπροσωπική επικοινωνία έχει διεισδύσει υπερβολικά στην προσωπική ζωή δημιουργώντας νέα εμπόδια και κινδύνους που περισσότερο δυσχεραίνουν, παρά καλύπτουν τις ανάγκες των χρηστών για επικοινωνία. Η παρούσα διπλωματική εργασία διερευνά τρόπους αντιμετώπισης του παραπάνω φαινομένου, μέσα από την ανάπτυξη μιας καινοτόμου εφαρμογής που στηρίζεται σε πληθοποριστικές πρακτικές. Η εφαρμογή θα εισάγει τους χρήστες της σε μια νέα εμπειρία, η οποία σκοπό έχει την ανάδειξη των δυνατοτήτων των χρηστών της, καθώς επίσης και την προβολή των σημερινών πολιτισμικών γεγονότων. Αυτό θα είναι εφικτό αξιοποιώντας την γνώση και την εμπειρία των χρηστών εντός της εφαρμογής για την άσκηση κριτικής, τον σχολιασμό και την συζήτηση με επίκεντρο εκδηλώσεις που αφορούν την πολιτισμική κοινότητα. Με αυτό τον τρόπο, αντιστρέφονται οι ρόλοι, καθώς η εφαρμογή παύει να αποτελεί τον πομπό όπως συνηθίζεται σήμερα, αλλά γίνεται ο δέκτης της πληροφορίας. Οι χρήστες με τη σειρά τους, θα αποτελούν την πηγή της πληροφορίας, χάρη στη γνώση που θα παρέχουν μέσα από τη χρήση της εφαρμογής. Αυτό θα οδηγήσει στη διαμόρφωση μιας πιο ενεργητικής στάσης απέναντι στα πολιτισμικά και κοινωνικά δρώμενα, με αποτέλεσμα την συμμετοχή του κοινωνικού συνόλου σε περισσότερα γεγονότα και εκδηλώσεις.
\newline
\indent
Η παραπληροφόρηση και η μονόπλευρη διαφήμιση στις μέρες μας, αφήνει ελάχιστα περιθώρια στους νέους που θέλουν να εξερευνήσουν την πόλη τους, να δοκιμάσουν νέες εμπειρίες και να ξεφύγουν από τον τετριμμένο τρόπο διασκέδασης. Ακόμη, η επιλογή ενός μέρους με βάση τις προτιμήσεις, μειώνει περισσότερο τις επιλογές καθώς οι ενδιαφερόμενοι επιλέγουν περισσότερο με βάση τι είναι σίγουρο και γνωστό, και λιγότερο με βάση τις προσωπικές τους επιθυμίες. Στόχος μας είναι η ανάπτυξη ενός καινοτόμου τρόπου γεφύρωσης του χάσματος που δημιουργούν οι σημερινοί ψηφιακοί κίνδυνοι, μέσα από τη χρήση σύγχρονων τεχνολογιών και πρωτοπόρων ιδεών. Η προσοχή μας θα επικεντρωθεί στη διαμόρφωση μιας εφαρμογής, της οποίας πρωταρχικός σκοπός είναι η διευκόλυνση της διαδικασίας της επικοινωνίας και της διασκέδασης με τη χρήση ενός νέου συστήματος μηνυμάτων σε τοποθεσίες επάνω σε χάρτη. Με αυτό τον τρόπο οι χρήστες θα μπορούν να πληροφορηθούν άμεσα για το τι συμβαίνει στην περιοχή ή το μέρος ενδιαφέροντος, να αλληλεπιδράσουν σε πραγματικό χρόνο με τους ευρισκόμενους την παρούσα χρονική στιγμή εκεί, να εξερευνήσουν νέα μέρη και να απολαύσουν τη διαδικασία της οργάνωσης της διασκέδασης. Πρωτίστως όμως, στόχος της εφαρμογής είναι να άρει τα σημερινά στερεότυπα για τις κοινωνικές ομάδες μεταξύ των νέων, φέρνοντας πιο κοντά το χρήστη με τα ενδιαφέροντά του. Να μετατρέψει σε πλειονότητες τις μειονότητες εκείνων που έχουν παραγκωνιστεί από το υπόλοιπο σύνολο λόγω των διαφορετικών αντιλήψεων και ενδιαφερόντων. Να ρίξει φως σε εκείνα τα μέρη που δεν παίρνουν την αντικειμενική αξία που τους αναλογεί. Να δώσει μάτια σε εκείνα τα άτομα  που επιλέγουν να ακολουθούν τυφλά την υπόλοιπη μάζα επειδή νομίζουν πως είναι μόνοι. Κάθε άτομο, οποιασδήποτε ομάδας, ηλικίας, γένους ή καταγωγής έχει δικαίωμα στον πολιτισμό και στην ενημέρωση. Δουλειά μας είναι να το κάνουμε ευρέως γνωστό.

\begin{keywords}
  Ανάπτυξη Εφαρμογών, Κινητή Εφαρμογή, Εξυπηρετητής, Βάση Δεδομένων, Πληθοπορισμός, \tl{React, React Native, JavaScript, Google Maps, Foursquare, RESTful APIs, Expo, Redux, MERN, Full Stack. }
\end{keywords}

\end{abstract}



\begin{abstracteng}
\tl{The problems of reaching communication today are gone. Thanks to the abundance of social media, one can search, chat or meet other people directly. However, this steep shift in interpersonal communication has penetrated too much into personal life, creating new barriers and dangers that make it harder to address users' needs for communication. This diploma thesis explores ways to address this phenomenon, through the development of an innovative application based on crowdsourcing techniques. The application will introduce its users to a new experience, which aims at highlighting the capabilities of its users, as well as the cultural events that take place today. This will be feasible by leveraging the users' knowledge and experience within the app for reviewing, commenting and discussing community-centered cultural events. In this way, the roles are reversed as the application ceases to be the transmitter of information as it is used to be today, but becomes the receiver of the information. Users, on the other hand, will be the source of this information, thanks to the knowledge they will provide through the application's interface. This will lead to a more active attitude towards cultural and social events.}

\tl{Disinformation and one-sided advertising nowadays leaves little room for young people to explore their city, try new experiences and escape the trivial way of having fun. In addition, choosing a preference-based part reduces ones choices, as people choose more on the basis of what is certain and familiar, and less on the basis of their personal wishes. Our goal is to present an innovative way to bridge the gap created by today's digital dangers through the use of modern technologies and pioneering ideas. Our focus will be the developing an application, whose primary purpose is to facilitate the communication and entertainment process by using a new messaging system on map locations. In this way, users can instantly learn about what's happening in their area or at the place of interest, interact in real time with those currently present at the event, explore new places, and enjoy the process of organizing their plans for having fun. Above all, however, the main goal of this application is to lift the current stereotypes for social groups among young people, bringing the user closer to their interests. Convert to minorities those who have been crowded out of the rest due to differences based on their interests. To shed light on those places that do not get the value they deserve. Give eyes to those people who choose to blindly follow the rest of the mass, because they think they are alone. Every person, of any age, gender or background, has a right into culture and information. Our job is to make it widely known.}



\begin{keywordseng}
    \tl{App Development, Mobile Application, Server, Database, Crowdsourcing, React, React Native, JavaScript, Node, Express, Mongo, Google Maps, Foursquare, RESTful APIs, Expo, Redux, MERN, Full Stack. }
\end{keywordseng}

\end{abstracteng}

\tableofcontents
\listoffigures
\listoftables

\clearpage
\renewcommand\lstlistlistingname{Κατάλογος Παραθέσεων}
\renewcommand\lstlistingname{\selectlanguage{greek}Παράθεση}
\lstlistoflistings


\chapter{Εισαγωγή}
\label{chap1}

%Εδώ αυτή κάνουμε μια γενική περιγραφή του χώρου εφαρμογής της διπλωματικής. Αναφέρουμε τα χαρακτηριστικά του χώρου και καταλήγουμε στα γενικότερα προβλήματα που αντιμετωπίζει ο χώρος. Η συζήτηση των προβλημάτων θα πρέπει να προϊδεάζει τον αναγνώστη για το τι θα προσπαθήσει να αντιμετωπίσει η διπλωματική, χωρίς ακόμα να αναφερόμαστε συγκεκριμένα στο αντικείμενο της διπλωματικής.
Είναι ευρεώς διαδεδομένο πως η ψηφιακή ενημέρωση βαδίζει πλέον με ραγδαίους ρυθμούς. Αμέτρητοι είναι σήμερα οι ιστότοποι και διαδικτυακές κοινότητες που σχετίζονται με όλες τις εκφάνσεις της καθημερινότητας, καθώς νέες εφαρμογές ενημέρωσης, οργάνωσης και επικοινωνίας αναδύονται στην ψηφιακή αγορά καθημερινά. Με μια αναζήτηση στο διαδίκτυο, ή με μια επισήμανση εντός της εφαρμογής, ο χρήστης μπορεί να ενημερωθεί για γεγονότα που τον ενδιαφέρουν, να δεχθεί ειδοποιήσεις για εκδηλώσεις που τον αφορούν, ή ακόμη και να προγραμματίσει το δρομολόγιό του, να υπολογίσει χρονικές και οικονομικές μεταβλητές και να σχεδιάσει το πλάνο του. Είναι λοιπόν ευνόητο το πρόβλημα που ανακύπτει από το παραπάνω φαινόμενο, σχετικά με την επικαιροποίηση και ενημέρωση της πληροφορίας που παρέχεται στο χρήστη. Παράγοντες όπως κυκλοφοριακή συμφόρηση, καιρικές αντιξοότητες, καθυστέρηση έναρξης μιας εκδήλωσης ή της άφιξης των συμμετεχόντων και άλλες αναπάντεχες εκβάσεις είναι αδύνατο να συνυπολογιστούν με κάποιον αλγόριθμο στις προαναφερθείσες εφαρμογές.

Από την άλλη, η απότομη στροφή των εφαρμογών και των μέσων κοινωνικής δικτύωσης γύρω από την ατομικότητα, έχει ως αντίκτυπο την αποδυνάμωση της συμμετοχής στο \textit{κοινωνικό γίγνεσθαι}. Πλέον είναι προτιμότερη η απομακρυσμένη έμμεση επικοινωνία με μέσα που τροφοδοτούν το χρήστη με εγωπαθή συμπτώματα και τον απομακρύνουν από την πραγμαματική έννοια της επικοινωνίας. Συχνό είναι επίσης το φαινόμενο εκμετάλλευσης της κοινωνικής προβολής για επαγγελματική ανέλιξη. Συνεπώς, γεγονότα  πολιτισμικού ενδιαφέροντος περνάνε σε δεύτερη μοίρα (βλ. Σχ \ref{socialmediausage}). 

Σαν αποτέλεσμα, τόποι κοινωνικού και πολιτισμικού περιεχομένου επισκιάζονται από αυτές τις εφαρμογές ``\textit{γίγαντες}'' (\selectlanguage{english}\textit{facebook, instagram, snapchat, twitter, tinder}\selectlanguage{greek} κλπ.). Εκδηλώσεις που χρήζουν προσοχής καταλήγουν να μην δέχονται την κατάλληλη προβολή. Είναι επομένως επιτακτική η ανάγκη ευαισθητοποίησης του χρήστη προκειμένου να δράσει υπερ του προσωπικού, αλλά ταυτοχρόνως και κοινωνικού οφέλους.   

\begin{figure}[!t]
	\includegraphics[scale=0.3]{figures/social-media-usage.png}
	\centering
	\caption{Οι χρήστες ξοδεύουν 14 φορές περισσότερο χρόνο χρησιμοποιώντας εφαρμογές όπως το \selectlanguage{english} \textit{facebook}\selectlanguage{greek}, έναντι εφαρμογών ενημέρωσης  (Πηγή: \cite{[AMP+14]})}
	\label{socialmediausage}
\end{figure}



\section{Αντικείμενο της διπλωματικής}

Εδώ αναφερόμαστε συγκεκριμένα στο τί θα κάνει η διπλωματική. Αναφέρουμε λεπτομερώς α) τα προβλήματα που θα λύσει (και που ήδη έχουν περιγραφεί γενικά στην προηγούμενη ενότητα), και β) πώς σκοπεύει να τα λύσει. 
Είναι σημαντικό κάποιος που θα διαβάσει την ενότητα αυτή να καταλάβει σε σημαντικό βαθμό τον σκοπό της διπλωματικής σας και τις τεχνικές δυσκολίες της, χωρίς να είναι αναγκαίο να δει όλα τα άλλα κεφάλαια. Η ενότητα αυτή θέλει πολύ προσοχή και καλύτερα να τη γράψετε αφού έχετε γράψει όλα τα υπόλοιπα κεφάλαια.

Το αντικείμενο με το οποίο καταπιάνεται το παρόν έργο, επικεντρώνεται στην σχεδίαση και την υλοποίηση μιας εφαρμογής, βασισμένη στην πρακτική ενεργοποίησης ενός ``\textit{πλήθους}'' ή μιας ομάδας, γνωστής και ως \textit{τακτική του πληθοπορισμού}. Η εφαρμογή που θα υλοποιηθεί θα υποστηρίζεται από \tl{iOS} πλατφόρμες σε όλες τις κινητές συσκευές. Η ανάπτυξη της εφαρμογής θα γίνει με τη χρήση των τελευταίων τεχνολογικών εργαλείων της αγοράς και θα ακολουθεί τις πιο σύγχρονες τάσεις που κυριαρχούν σήμερα στον ψηφιακό κόσμο.
\newline
\indent
 Πρωταρχικός στόχος της παρούσας εργασίας είναι να αναδείξει τις σύγχρονες τεχνολογίες που χρησιμοποιούνται σήμερα για την ανάπτυξη εφαρμογών, μέσα από τη σχεδίαση και υλοποίηση μιας καινοτόμου εφαρμογής. Ταυτοχρόνως, εισάγονται νέες ιδέες και τρόποι ανάδειξης τόπων πολιτισμικού περιεχομένου, μέσα από την ενεργοποίηση του κοινωνικού συνόλου. Οι χρήστες συμμετέχουν άμεσα σε πολιτισμικά δρώμενα, αναλαμβάνοντας ρόλους που έχουν άμεσο αντίκτυπο στην εικόνα του πολιτισμού. Μέσα από τη διαμόρφωση σχέσεων μεταξύ των χρηστών και της πολιτισμικής κληρονομιας, η εφαρμογή συνδράμει τόσο στην εξέλιξη της κοινωνίας, όσο και του πολιτισμού. 

\subsection{Συνεισφορά}

Η παρούσα διπλωματική εργασία σκοπό έχει να εμπλουτίσει τις δυνατότητες των χρηστών με πολιτιστικά ενδιαφέροντα. Χρησιμοποιώντας τις πιο πρόσφατες τεχνολογίες, θα προσφέρει μια ολοκληρωμένη εμπειρία μέσα από μια πλήρως λειτουργική και σύγχρονη εφαρμογή για πλατφόρμες \tl{iOS}. Η εφαρμογή θα δίνει στους χρήστες τη δυνατότητα να ενημερώνονται σχετικά με γεγονότα και εκδηλώσεις πολιτισμικού περιεχομένου, μέσα από τεχνικές βασισμένες στον πληθοπορισμό. Πρωταρχικό μέλημα είναι η προαγωγή του πολιτισμού, αξιοποιώντας τις δυνατότητες του διαδικτύου προς όφελος των χρηστών. Η λειτουργία της εφαρμογής θα σέβεται και θα προασπίζει την ασφάλεια του χρήστη, χωρίς να εμποδίζει την αλληλεπίδρασή του με το υπόλοιπο σύνολο, μέσα από την ανταλλαγή εντυπώσεων και το σχολιασμό γεγονότων. Επίσης, οι λειτουργικότητες της εφαρμογής μπορούν να αξιοποιηθούν από πολιτισμικούς οργανισμούς και φορείς για την βελτίωση του ρόλου τους μέσα από ανατροφοδότηση των χρηστών για μελλοντικούς σχεδιασμούς. 
\newline
\indent
Είναι κοινώς παραδεκτό πως ο πολιτισμός μπορεί να λειτουργήσει ως πεδίο σεβασμού της ετερότητας. H εφαρμογή μπορεί να διευκολύνει στο πεδίο αυτό, αναδεικνύοντας την αξία περιθωριοποιημένων ομάδων. Με τη δημιουργία μιας κοινής διεπιφάνειας χάρτη, ενοποιεί τις κοινωνικές ομάδες με κοινά ενδιαφέροντα και αναδεικνύει κοινούς τόπους. Συνδέει άτομα άγνωστα μέχρι πρότινος, άροντας όλα τα εμπόδια στην επικοινωνία και προωθώντας την διαλεκτικότητα, την αμοιβαιότητα και τον σεβασμό μνημείων πολιτισμικού ενδιαφέροντος. Επιπρόσθετα συμβάλλει στη δημιουργία μιας κουλτούρας του διαδυκτίου που σκοπό έχει την αναβάθμιση των χρηστών και όχι το κέρδος ή την εμπορευματοποίηση. 



\section{Οργάνωση του τόμου}
Η δομή της παρούσας εργασίας είναι οργανωμένη σε έξι επί μέρους κεφάλαια:

\begin{enumerate}
\item Στο \textbf{2ο Κεφάλαιο} παρουσιάζονται οι σημαντικότερες έννοιες, θεωρητικές αλλά και τεχνικές, οι οποίες είναι άρρηκτα συνδεδεμένες με το έργο και θα βοηθήσουν στην διαμόρφωση μιας σφαιρικής εικόνας. Επιπλέον παρουσιάζει σχετικές προσπάθειες που έχουν γίνει στο παρελθόν για συναφή αντικείμενα και θα βοηθήσουν στην ανάπτυξη της εφαρμογής. Τέλος, γίνεται αναφορά στις βασικές τεχνολογικές έννοιες που θα αξιοποιηθούν στην εφαρμογή, όπως \tl{REST APIs, OAuth, node} κλπ.
\item Στο \textbf{3ο Κεφάλαιο} αναλύονται οι λειτουργικές και μή λειτουργικές απαιτήσεις του συστήματος.Επίσης παρουσιάζεται η αρχιτεκτονική που θα ακολουθηθεί κατά την ανάπτυξη του συστήματος των διεπιφανειών του χρήστη, του εξυπηρετητή αιτημάτων και της βάσης δεδομένων.  
\item Στο \textbf{4ο Κεφάλαιο} επεξηγούνται οι τεχνικές σχεδίασης των διεπιφανειών της εφαρμογής και γίνεται μια εκτεταμένη ανάλυση του τρόπου σχεδίασης της κάθε οθόνης ξεχωριτσά. Επίσης παρατίθενται η λογική σχεδίασης του εξυπηρετητή αιτημάτων και της βάσης δεδομένων.
\item Στο \textbf{5ο Κεφάλαιο} αναλύεται λεπτομερώς η υλοποίηση του συστήματος. Αρχικά, γίνεται λόγος για τις λειτουργικότητες της εφαρμογής και τις τεχνικές με τις οποίες αυτές υλοποιούννται. Στη συνέχεια, γίνεται αναφορά στο τρόπο διαχείρησης των δεδομένων εντός της εφαρμογής. Τέλος, επεξηγούνται οι λόγοι επιλογής της πλατφόρμας προγραμματισμού. 
\item Στο \textbf{6ο Κεφάλαιο} γίνεται μια συνοπτική παρουσίαση του τελικού συστήματος, καθώς επίσης και μια ανάλυση των λειτουργικοτήτων που θα μπορούσαν να προστεθούν μελλοντικά στην εφαρμογή.  
\end{enumerate}





\chapter{Θεωρητικό και Τεχνολογικό υπόβαθρο}
\label{chap3}

Σε αυτό το κεφάλαιο θα γίνει μια εκτεταμένη ανάλυση όλων των βασικών εννοιών που θεμελιώνουν τόσο τη θεωρητική, όσο και την τεχνολογική βάση στην οποία στηρίζεται η διπλωματική εργασία. Καθώς το παρόν έργο αποτελεί συγκερασμό δύο επιστημονικών κλάδων (Κοινωνιολογία και Πληροφορική), είναι αναγκαία η διαίρεση αυτού του κεφαλαίου σε ΠΟΣΕΣ ενότητες. Στην ενότητα 3.1 αναλύεται το θεωρητικό μοντέλο και οι κοινωνιολογικές έννοιες που το συνιστούν. Η ενότητα 3.2 αναφέρεται σε σχετικές ερευνητικές προσπάθειες που έχουν προηγηθεί. Τέλος, στην ενότητα 3.3 παρατίθενται τα σχετικά τεχνολογικά εργαλεία που χρησιμοποιήθηκαν.

\section{Θεωρητικό Υπόβαθρο - Βασικές Έννοιες}

\subsection{Κοινωνικός Ρόλος των Σύγχρονων Εφαρμογών}
Πρωταρχικό ρόλο στην επιτυχία μιας εφαρμογής παίζει το κίνητρο με το οποίο αυτή εξαφαλίζει τη διαρκή ενασχόληση του χρήστη. Βασικό στοιχείο για να επιτευχθεί αυτό είναι o κοινωνικός ρόλος που επωμίζεται ο χρήστης εντός της εφαρμογής. Oι σημερινές εφαρμογές έχουν στρέψει την προσοχή γύρω από την κοινωνική προβολή και επιβολή, απομακρύνοντας την προσοχή από δημοσιεύσεις που αφορούν κοινωνικά δρώμενα, τεχνολογικά επιτεύγματα και γενικότερα τον συλλογικό βίο (βλ. Σχ. \ref{socialsharing}). Έτσι, παρατηρείται η προσκόλληση στα μέσα κοινωνικής δικτύωσης ως τρόπο άσκησης κοινωνικής επιρροής. Δημιουργούνται συνεπώς εγωκεντρικές τάσεις που τροφοδοτούν νέες ανάγκες και οδηγούν σε νέες ομοειδείς εφαρμογές. Τέτοια παραγείγματα είναι η ανάγκη για δημοτικότητα και κοινωνική αποδοχή \cite{[IND+16]} από τρίτους, η έντονη εμμονή με την προσωπική εικόνα στον ψηφιακό κόσμο και η ενίσχυση των απρόσωπων σχέσεων \cite{[JAR+18]}. 

\begin{figure}[H]
    \includegraphics[scale=0.3]{figures/social-share-has-decreased.png}
    \centering
    \caption{Η δημοσιεύσεις κοινωνικών γεγονότων έχουν υποχωρήσει αισθητά. (Πηγή: \cite{[VEN+18]})}
    \label{socialsharing}
\end{figure}


Προκειμένου να επαναπροσδιοριστεί ο ρόλος των εφαρμογών, είναι απαραίτητο να αναθεωρηθούν τα κίνητρα με τα οποία αυτές κεντρίζουν το ενδιαφέρον του χρήστη. Η κοινωνική επίδειξη, να αντικατασταθεί με την κοινωνική συνεισφορά, ενώ η απομόνωση θα πρέπει να δώσει τη θέση της στην επανένταξη του ατόμου στο κοινωνικό σύνολο. Η ενημέρωση μέσω των εφαρμογών, οφείλει να έχει κοινωνικό χαρακτήρα και όχι να προβάλλει την προσωπική ζωή, ή να ωθεί σε κοινωνικά σύνδρομα τους χρήστες \cite{[BBC+18]}. 

\subsection{Ανάγκη της Κοινωνικής Προσφοράς}
Έχοντας υπόψη τα παραπάνω, καταλήγει κανείς εύκολα στο συμπέρασμα ότι υπάρχει μεγάλη ανάγκη να επανασυνδεθεί ο ρόλος των κοινωνικών εφαρμογών με την συνεισφορά για το κοινό συμφέρον. Αν και ζούμε σε μία εποχή όπου η τεχνολογία προχωράει με αλματώδεις ρυθμούς, ελάχιστο είναι το ποσοστό του συνόλου που γνωρίζει για τα επιτεύγματα των συγχρόνων του στους διάφορους επιστημονικόυς τομείς. Ακόμη κι αν το άτομο εκδηλώνει ενδιαφέρον, είναι δύσκολο να ενημερωθεί όταν όλες οι θεματικές περιστρέφονται γύρω από την προσωπική ζωή. Προκύπτει, λοιπόν μια νέα ανάγκη για κινητοποίηση του χρήστη να αλληλεπιδράσει με το κοινωνικό σύνολο. Αυτό είναι εφικτό χρησιμοποιώντας τις δυνατότητες των ήδη γνωστών εφαρμογών, αυτή τη φορά με σκοπό την ενημέρωση για γεγονότα που αφορούν την ευρύτερη πολιτισμική κοινότητα. 

\subsection{Η Έννοια του Πληθοπορισμού}
%Εδώ γράφουμε σύντομα τις τεχνικές/μεθοδολογίες/μοντέλα που πιθανά θα χρησιμοποιήσει η διπλωματική και είναι αναγκαία η κατανόησή τους από τον αναγνώστη πριν από την παρουσίαση της ανάλυσης και σχεδίασης του συστήματος. Πρόκειται για τεχνικές/μεθοδολογίες/μοντέλα που έχουν προταθεί από άλλους και δεν είναι πρωτότυπη δουλειά της διπλωματικής. Μετά βάζουμε μία ενότητα για κάθε τεχνική/μεθοδολογία/μοντέλο, όπου και δίνουμε λεπτομερή περιγραφή.
Συνδυάζοντας τη δύναμη της συλλογικής προσφοράς με την τεχνολογία και τεχνογνωσία που είναι διαθέσιμη σήμερα, η ενημέρωση μπορεί να πάρει νέα διάσταση. Με στοχευμένο προσανατολισμό της κοινωνικής διάθεσης για δράση προς μία συγκεκριμένη κατεύθυνση, η κοινωνία μπορεί να στρατολογήσει τα ίδια τα μέλη της προκειμένου να προάγει τις τεχνολογικές, πολιτισμικές και κοινωνικές εκδηλώσεις, να συντονίσει την πληροφορία και να ενημερώσει το σύνολο. Σύμφωνα με τη \selectlanguage{english}\textit{New York Times}\selectlanguage{greek}, χάρις στην αυξανόμενη συνδεσιμότητα μέσω του διαδικτύου, εκατομμύρια ανθρώπων μπορούν να συνεισφέρουν ιδέες και πληροφορίες για προβλήματα οποιασδήποτε μορφής. Η πρακτική της συμμετοχής ενός «πλήθους» ή μιας ομάδας για έναν κοινο στόχο ή επίλυση κοινών προβλημάτων, με την επιστράτευση της τεχνολογίας ως διαύλου επικοινωνίας ονομάζεται \textit{πληθοπορισμός} (\selectlanguage{english}\textit{Crowdsourcing})\selectlanguage{greek} \cite{[CSW+18]}.

\subsubsection{Εφαρμογές του Πληθοπορισμού}
Τα πεδία στα οποία μπορεί να αξιοποιηθεί η τεχνική του πληθοποριμού είναι αμέτρητα. Ο,τιδήποτε μπορεί να αποκτήσει συνεργατικό χαρακτήρα. Αναφορικά, παρατίθενται μερικοί κλάδοι όπου ανθεί η τεχνική αυτή:
\begin{description}[font=$\bullet$~\normalfont\color{black}]
\item [Εκπαίδευση]
\item [Οικονομία]
\item [Επιστήμη και Υγεία]
\item [ΙΤ]
\item [Διαφήμιση]
\item [Επιχειρηματικότητα]
\item [Κοινωνικές Εκδηλώσεις και ΜΚΟ]
\end{description}

\subsubsection{Μοντέλο Πληθοπορισμού στην Εφαρμογή}
Η εφαρμογή σκοπεύει να χρησιμοποιήσει την πρακτική του πληθοπορισμού για να συγκεντρώσει πληροφορίες σχετικές με πολιτισμικές και κοινωνικές εκδηλώσεις. Οι χρήστες θα έχουν τη δυνατότητα να αξιολογούν τα πάντα γύρω από ένα γεγονός. Θα είναι δυνατή η ενημέρωση για τυχόν αλλαγές της ώρας και του τόπου, για προβλήματα που μπορεί να δυσκολέψουν την διεξαγωγή της εκδήλωσης, ή ιδέες για την καλυτέρευση αυτής. Το σημαντικό στοιχείο όλων των παραπάνω είναι πως θα μπορούν να γίνουν σε πραγματικό χρόνο. Αυτό θα έχει σαν αποτέλεσμα την καλύτερη αλληλεπίδραση μεταξύ συμμετεχόντων και διοργανωτών, την αποφυγή προβλημάτων και παρανοήσεων και την καλύτερη ενημέρωση του πλήθους. Διοργανωτές και συμμετέχοντες, θα μπορούν να εκφράσουν τη γνώμη τους, όλοι ως χρήστες της εφαρμογής. Η πληροφορία θα επικαιροποιείται διαρκώς από όσους παρευρίσκονται ήδη εκεί και θα διαδίδεται σε όσους μέχρι τώρα την αγνοούσαν.Το παραπάνω μοντέλο υλοποιειται μέσω ενός συστήματος μηνυμάτων μεταξύ των χρηστών. 

\section{Σχετικές Ερευνητικές Προσπάθειες και Πρότυπα}
Αν και το παρόν έργο αποτελεί προσωπική πρωτοβουλία, η ιδέα της διπλωματικής έχει κάποιες επιρροές και από άλλα παράλληλα έργα στον ίδιο τομέα. Τέτοια έργα είναι το \selectlanguage{english}\textit{WITHcrowd}\selectlanguage{greek} της ερευνητικής ομάδας του εργαστηρίου Ευφυών Συστημάτων (\selectlanguage{english}\textit{ISLAB}\selectlanguage{greek}) του Εθνικού Μετσόβιου  Πολυτεχνείου, που αποτελεί μέρος του ευρωπαϊκού προγράμματος \selectlanguage{english}\textit{WITH}\selectlanguage{greek}. Το έργο αυτό αποτέλεσε πηγή έμπνευσης για το κομμάτι του πληθοπορισμού στην εφαρμογή της παρούσας εργασίας. Χρησιμοποιήθηκαν επίσης to πρότυπo ανοιχού κώδικα \selectlanguage{english}\textit{Foursquare API}\selectlanguage{greek}, και οι διεπαφές αυτού,  \selectlanguage{english}\textit{Foursquare Autocomplete}\selectlanguage{greek} και \selectlanguage{english}\textit{Foursquare Places}\selectlanguage{greek}. Συνολικά, η εφαρμογή είναι μια καινοτομία η οποία προσπαθεί να στρέψει ήδη υπάρχουσες πρακτικές προς μια νέα κατεύθυνση, χρησιμοποιώντας τις σύγχρονη τεχνογνωσία για έναν πρωτοποριακό σκοπό.   

\subsection{\selectlanguage{english}WITHcrowd}
To \selectlanguage{english}WITHcrowd\cite{[WIT+18]}\selectlanguage{greek} είναι μια πρωτοβουλία της ευρωπαϊκής κοινότητας για τη συλλογή και ταξινόμηση δεδομένων πολιτισμικού περιεχομένου. Αποτελείται από μία πλατφόρμα που εκθέτει τις διεπαφές (ΑΡΙ) διαφόρων πυλών (\selectlanguage{english}portals)\selectlanguage{greek} και ψηφιακών αποθηκών (\selectlanguage{english}repositories)\selectlanguage{greek}. Eπιτρέπει στους χρήστες την αναζήτηση ψηφιακού περιεχομένου από μια σειρά διαφορετικών και ανεξάρτητων αποθετηρίων και βάσεων δεδομένων από ένα ενιαίο σημείο πρόσβασης. Τα αποθετήρια που μπορούν να αναζητηθούν περιλαμβάνουν μεταξύ άλλων την \selectlanguage{english}Europeana,\selectlanguage{greek} την Ψηφιακή Δημόσια Βιβλιοθήκη της Αμερικής, το \selectlanguage{english}YouTube,\selectlanguage{greek} το Μουσείο \selectlanguage{english}Rijks,\selectlanguage{greek} την Εθνική Βιβλιοθήκη της Αυστραλίας και την Ψηφιακή Νέα Ζηλανδία. Όλη η δύναμη της πλατφόρμας συγκεντρώνεται στο γεγονός ότι ο χρήστης είναι υπεύθυνος για την επίτευξη του στόχου του προγράμματος. Αυτή την πρακτική επιχειρεί να υιοθετήσει η εφαρμογή που θα αναλυθεί στα επόμενα κεφάλαια. 

\subsection{\selectlanguage{english}Foursquare API}
Το \selectlanguage{english}\textit{Foursquare API}\cite{[4SQ+18]}\selectlanguage{greek} είναι η διεπαφή της εφαρμογής \selectlanguage{english}\textit{Foursquare}\selectlanguage{greek} και παρέχεται στην κοινότητα της πληροφορικής δωρεάν. Δίνει τη δυνατότητα στους προγραμματιστές να χρησιμοποιήσουν δεδομένα που αφορούν τελικά σημεία (\selectlanguage{english}endpoints)\selectlanguage{greek} σε Ενιαίους Εντοπιστές Πόρων (\selectlanguage{english}URLs)\selectlanguage{greek}, όπως τα στοιχεία μιας υπηρεσίας (πχ. τοποθεσία, πληροφορίες επικοινωνίας, διεύθυνση, όνομα, κατηγορία, ώρες λειτουργίας, παροχές κλπ).

To \selectlanguage{english}\textit{Foursquare}\selectlanguage{greek} έφερε την επανάσταση στα μέσα κοινωνικής δικτύωσης με την καινοτομία του ``\textit{\selectlanguage{english}check in}''.  Η λειτουργία της εφαρμογής \selectlanguage{english}Foursquare\selectlanguage{greek} συνοψίζεται στην αξιολόγηση και άσκηση κριτικής σε κέντρα διασκέδασης, εστιατόρια και χώρους ψυχαγωγίας. Ο χρήστης δημοσιεύει την τοποθεσία του κάνοντας \selectlanguage{english}check in\selectlanguage{greek} και ενημερώνει τους υπόλοιπους χρήστες-φίλους του. Η εφαρμογή που θα υλοποιηθεί κάνει μια απόπειρα να διοχετεύσει την πληροφορία σε αντίστοιχα μέρη πολιτισμικού ή κοινωνικού περιεχομένου και να την αξιοποιήσει για την αξιολόγησή τους.


\section{\selectlanguage{greek}Τεχνολογικό υπόβαθρο - Βασικές Έννοιες}
Στη συνέχεια θα γίνει μια εισαγωγή στις τεχνολογίες και τα εργαλεία προγραμματισμού. Έπειτα ακολουθεί η ανάλυση των σύγχρονων τεχνολογικών μέσων που αποτελούν τη βάση των μεθόδων και μοντέλων που χρησιμοποιήθηκαν στην υλοποίηση της εφαρμογής. 

Όταν ένας προγραμματιστής αναπτύσσει μία εφαρμογή, καλείται να προσδιορίσει τρία πράγματα:
\begin{enumerate}
\item Tην φύση της εφαρμογής - αν θα τρέχει σε φυλλομετρητή (\selectlanguage{english}Web Application)\selectlanguage{greek} ή αν θα είναι μητρική (\selectlanguage{english}Native Application)\selectlanguage{greek}.
\item Την γλώσσα στην οποία θα αναπτύξει την εφαρμογή (\selectlanguage{english}Programming Language)\selectlanguage{greek}
\item Την πλατφόρμα λογισμικού στην οποία θα υλοποιήσει την εφαρμογή (\selectlanguage{english}\textit{SDK})\selectlanguage{greek}
\end{enumerate}

\subsection{Είδη Γλωσσών Προγραμματισμού}
Πρωτού μιλήσουμε για τα είδη των σύγχρονων εφαρμογών, είναι απαραίτητο να αναλύσουμε τις κατηγορίες των αντίστοιχων γλωσσών και τις διακρίσεις μεταξύ αυτών. Μια γλώσσα μπορεί να ανήκει σε μία από τις παρακάτω κατηγορίες:
\begin{description}[font=$\bullet$~\normalfont\color{black}]
\item [Μετταγλωτισμένες ή Μητρικές γλώσσες (\selectlanguage{english}Compiled or Native Languages)]\selectlanguage{greek}
\item [Διαχειριζόμενες Γλώσσες (\selectlanguage{english}Managed Languages)]\selectlanguage{greek}
\item [Δυναμικές Γλώσσες (\selectlanguage{english}Dynamic Languages)]\selectlanguage{greek}
\end{description}Η μητρική γλώσσα (\selectlanguage{english}native language)\selectlanguage{greek} είναι μια γλώσσα που μπορεί να τρέξει στην πλατφόρμα του λειτουργικού συστήματος χωρίς να μετατραπεί σε άλλη μορφή κώδικα από τους μεταγλωττιστές (\selectlanguage{english}compilers)\selectlanguage{greek}. Αυτό σημαίνει πως η υλοποίησή της συνοψίζεται κυρίως στη χρήση μεταγλωττιστών (\selectlanguage{english}compilers)\selectlanguage{greek}, οι οποίοι είναι υπεύθυνοι για τη μετατροπή του κώδικα από γλώσσα μηχανής σε πηγαίο κώδικα (πχ. \selectlanguage{english}\textit{C++, C\#, Java, Swift})\selectlanguage{greek}. Η διαχειριζόμενη γλώσσα (\selectlanguage{english}managed language)\selectlanguage{greek} είναι μια γλώσσα που πρέπει να μετατραπεί ή να ερμηνευτεί πριν να εκτελεστεί στην πλατφόρμα (πχ.\selectlanguage{english} \textit{.NET})\selectlanguage{greek}. Σε αυτή την περίπτωση, ο κώδικας θα εκτελεστεί υπό τη διαχείριση μιας εικονικής μηχανής γλώσσας κοινού χρόνου εκτέλεσης ή, όπως είναι γνωστή,\selectlanguage{english} \textit{CLR}\selectlanguage{greek} \cite{[STR+09],[DEV+03]}. Η δυναμική γλώσσα προγραμματισμού (\selectlanguage{english}dynamic language)\selectlanguage{greek} είναι μια κλάση αποτελούμενη από γλώσσες υψηλού επιπέδου (\selectlanguage{english}high-level programming languages)\selectlanguage{greek}, οι οποίες έχουν την ιδιότητα να εκτελούν πολλαπλές εντολές κατά το στάδιο της εκτέλεσης, σε αντίθεση με τις υπόλοιπες γλώσσες που εκτελούν εντολές στο στάδιο μεταγλώττισης  (πχ.\selectlanguage{english} \textit{Python, JavaScript, PHP, Ruby, MATLAB, Elixir})\selectlanguage{greek} \cite{[MIC05], [ADV09]}.

Οι γλώσσες υψηλού επιπέδου, όπως η \selectlanguage{english}Python\selectlanguage{greek} και η \selectlanguage{english}Ruby\selectlanguage{greek}, έχουν αποκτήσει μεγάλη απήχηση τα τελευταία χρόνια. Για πολλούς, αποτελούν τη νέα γεννιά γλωσσών προγραμματισμού. Ωστόσο οι ρίζες τους ανατρέχουν στις αρχές της δεκαετίας του '50, με την γέννηση της \selectlanguage{english}Lisp\selectlanguage{greek} της πρώτης γλώσσας υψηλού επιπέδου. Πλέον, οι δυναμικές γλώσσες συναντόνται τόσο σε πραγματικά διαδικτυακά συστήματα, όσο και σε εφαρμογές που πριν κυριαρχούσαν οι στατικές γλώσσες προγραμματισμού \cite{[ADV09]}.

Εμείς θα επικεντρωθούμε στην πρώτη κατηγορία, των μητρικών ή αλλιώς μεταφρασμένων γλωσσών προγραμματισμού. Το πλεονέκτημα χρήσης μητρικού κώδικα (\selectlanguage{english}native code\selectlanguage{greek}) βρίσκεται στο γεγονός ότι προγράμματα σε τέτοιες γλώσσες είναι γρηγορότερα, χάρις τον μειωμένο χρόνο επιβάρυνσης (\selectlanguage{english}overhead)\selectlanguage{greek} της διαδικασίας μετάφρασης. Αυτό οφείλεται στην ιδιότητά τους να μεταγλωττίζονται κατά το χρόνο μεταγλώττισης και όχι κατά το χρόνο εκτέλεσης, όπως συμβαίνει με άλλα προγράμματα.

Οι γλώσσες χαμηλού επιπέδου (\selectlanguage{english}low-level programming languages)\selectlanguage{greek} είναι δομημένες έτσι ώστε να υφίστανται την τυπική μεταγλώττιση, ειδικά όταν η αποτελεσματικότητα είναι πρωταρχικό μέλημα, έναντι της μεταγλώττισης που υποστηρίζει πολλές πλατφόρμες (\selectlanguage{english}cross-platform)\selectlanguage{greek}. Για αυτές τις γλώσσες, υπάρχουν περισσότερες ένα-προς-ένα αντιστοιχίες ανάμεσα στο προγραμματισμένο κώδικα και τις λειτουργίες υλικού που εκτελούνται σε κώδικα μηχανής, καθιστώντας ευκολότερο για τους προγραμματιστές τον έλεγχο χρήσης της κεντρικής μονάδας επεξεργασίας και μνήμης με λεπτομέρεια. 
Με κάποια προσπάθεια, είναι πάντα δυνατή η σύνταξη μεταγλωττιστών ακόμα και για παραδοσιακά ερμηνευμένες γλώσσες. Για παράδειγμα, η\selectlanguage{english} Common lisp\selectlanguage{greek} μπορεί να μεταγλωττιστεί σε \selectlanguage{english}Java bytecode\selectlanguage{greek} (στη συνέχεια ερμηνεύεται από την εικονική μηχανή\selectlanguage{english} Java\selectlanguage{greek}), σε κώδικα\selectlanguage{english} C\selectlanguage{greek} (στη συνέχεια, μεταγλωττίζεται στον εγγενή κώδικα μηχανής) ή απευθείας σε εγγενή κώδικα. Οι γλώσσες προγραμματισμού που υποστηρίζουν πολλαπλούς στόχους σύνταξης δίνουν μεγαλύτερο έλεγχο στους προγραμματιστές για να επιλέξουν είτε ταχύτητα εκτέλεσης είτε συμβατότητα μεταξύ πλατφόρμων \cite{[SQA+07]}.

\selectlanguage{english}
\begin{lstlisting}[language=Python, caption=\selectlanguage{greek}Παράδειγμα κώδικα σε \selectlanguage{english}Python]
import numpy as np
 
def incmatrix(genl1,genl2):
    m = len(genl1)
    n = len(genl2)
    M = None #to become the incidence matrix
    VT = np.zeros((n*m,1), int)  #dummy variable
 
    #compute the bitwise xor matrix
    M1 = bitxormatrix(genl1)
    M2 = np.triu(bitxormatrix(genl2),1) 
 
    for i in range(m-1):
        for j in range(i+1, m):
            [r,c] = np.where(M2 == M1[i,j])
            for k in range(len(r)):
                VT[(i)*n + r[k]] = 1;
                VT[(i)*n + c[k]] = 1;
                VT[(j)*n + r[k]] = 1;
                VT[(j)*n + c[k]] = 1;
 
                if M is None:
                    M = np.copy(VT)
                else:
                    M = np.concatenate((M, VT), 1)
 
                VT = np.zeros((n*m,1), int)
 
    return M
\end{lstlisting}

\selectlanguage{greek}
\subsection{Μητρικές Γλώσσες Προγραμματισμού (\selectlanguage{english}Native Programming Languages)}
\selectlanguage{greek}
%Μιλάμε για τις δυνατοτητες τους και τα μειονεκτηματα τους (αναφορα σε σχετικα αρθρα, παραθεση δειγματος κωδικα και παραθεση διαγραμματων με την πτωση της χρησης τους)
Έχοντας προσδιορίσει τις κατηγορίες γλωσσών προγραμματισμού, μπορεί κανείς επομένως να αντιληφθεί την ανάγκη των μητρικών ή αλλιώς μεταγλωττισμένων γλωσσών για την ανάπτυξη εφαρμογών σε συγκεκριμένες πλατφόρμες. Αυτές οι γλώσσες είναι κατά κύριο λόγο σχεδιασμένες να τρέχουν σε μια ορισμένη πλατφόρμα για καλύτερη απόδοση και ευκολότερο σχεδιασμό. Τέτοιες γλώσσες που χρησιμοποιούνται σήμερα στις εφαρμογές κινητών συσκευών είναι η \selectlanguage{english}Swift\selectlanguage{greek} και η \selectlanguage{english}Java\selectlanguage{greek}. Η πρώτη χρησιμοποιείται για την υλοποίηση εφαρμογών σε \selectlanguage{english}iOS\selectlanguage{greek} πλατφόρμες, ενώ η δεύτερη για την ανάπτυξη εφαρμογών που τρέχουν σε \selectlanguage{english}Android\selectlanguage{greek} πλατφόρμες.

\subsubsection{\selectlanguage{english}Swift - iOS Development}
\selectlanguage{greek}
Η \selectlanguage{english}\textit{Swift}\selectlanguage{greek} είναι μια μεταγλωττισμένη, γενικού σκοπού, πολυπαραδειγματική γλώσσα προγραμματισμού που έχει αναπτυχθεί από την \selectlanguage{english}\textit{Apple Inc.}\selectlanguage{greek} για τα προϊόντα τις ίδιας εταιρείας (\selectlanguage{english}\textit{iOS, macOS, watchOS, tvOS}\selectlanguage{greek}). Είναι σχεδιασμένη ώστε να δουλεύει με το \textit{\selectlanguage{english}XCode IDE\selectlanguage{greek}} και χρησιμοποιείται από την κοινότητα των \selectlanguage{english}iOS developers\selectlanguage{greek} για ανάπτυξη \selectlanguage{english}iOS\selectlanguage{greek} εφαρμογών, υποστηριζόμενων από τις πλατφόρμες και τα προϊόντα \selectlanguage{english}Apple\selectlanguage{greek} \cite{[SWIFT1+16]}.

Τα πρώτα βήματα για την δημιουργία της \selectlanguage{english}Swift\selectlanguage{greek} ξεκίνησαν υπό την καθοδήγηση του \selectlanguage{english}C.~Lattner\selectlanguage{greek}, εντός της \selectlanguage{english}Apple\selectlanguage{greek}. Με επιρροές από γλώσσες όπως οι \selectlanguage{english}\textit{Objective-C, Rust, Haskell, C\#}\selectlanguage{greek} και αρκετές ακόμη \cite{[SWIFT2+14]}, η \selectlanguage{english}Swift\selectlanguage{greek} ήρθε στο προσκήνιο επίσημα για πρώτη φορά στο Παγκόσμιο Συνέδριο Προγραμματιστών (\selectlanguage{english}WWDC\selectlanguage{greek}) τον Ιούνιο του 2014 \cite{[SWIFT3+14]}. 

Βασικά γνωρίσματα αυτής της γλώσσας είναι η αντικειμενοστρέφεια (\selectlanguage{english}\textit{OO} language\selectlanguage{greek}), και η απλουστευμένες δομές. Η \selectlanguage{english}Swift\selectlanguage{greek} βασίζεται σε θεωρητικές έννοιες των μοντέρνων γλωσσών προγραμματισμού και προσπαθεί να παρουσιάσει μια απλούστερη συντακτική προσέγγιση \cite{[SWIFT4], [SWIFT5]}. 

\selectlanguage{english}
\begin{lstlisting}[language=Swift, caption=\selectlanguage{greek}Παράδειγμα κώδικα σε \selectlanguage{english}Swift]
    import UIKit
    import AVFoundation
    
class ViewController: UIViewController {
    
    @IBOutlet weak var darkBlueBG: UIImageView!
    @IBOutlet weak var powerBtn: UIButton!
    @IBOutlet weak var cloudHolder: UIView!
    @IBOutlet weak var rocket: UIImageView!
    @IBOutlet weak var hustleLbl: UILabel!
    @IBOutlet weak var onLbl: UILabel!
    
    var player: AVAudioPlayer!
    
    override func viewDidLoad() {
        super.viewDidLoad()
        
        let path = Bundle.main.path(forResource: "hustle-on", ofType: "wav")!
        let url = URL(fileURLWithPath: path)
        do {
            player = try AVAudioPlayer(contentsOf: url)
            player.prepareToPlay()
        } catch let error as NSError {
            print(error.description)
        }
    }
    
    @IBAction func powerBtnPressed(_ sender: Any) {
        cloudHolder.isHidden = false
        darkBlueBG.isHidden = true
        powerBtn.isHidden = true
        
        player.play()
        
        UIView.animate(withDuration: 2.3, animations: {
            self.rocket.frame = CGRect(x: 0, y: 140, width: 375, height: 402)
        }) { (finished) in
            self.hustleLbl.isHidden = false
            self.onLbl.isHidden = false
        }
    }
}

\end{lstlisting}


\subsubsection{\selectlanguage{english}Java - Android Development}
\selectlanguage{greek}
Η \selectlanguage{english}\textit{Java}\selectlanguage{greek} είναι γενικού σκοπού, ταυτοχρονισμένη (\selectlanguage{english}concurrent\selectlanguage{greek}), βασισμένη σε κλάσεις (\selectlanguage{english}class-based\selectlanguage{greek}) και αντικειμενοστραφής (\selectlanguage{english}\textit{OO}\selectlanguage{greek}) γλώσσα \cite{[JAVA1]}. Έχει σχεδιαστεί ειδικά για να επιτρέπει όσο το δυνατό μεγαλύτερη ανεξαρτησία στον προγραμματιστή και ευελιξία όσον αφορά τη συγγραφή κώδικα με συνδυασμό πολλών βιβλιοθηκών. Ακολουθεί τη νοοτροπία ``\selectlanguage{english}\textit{WORA}\selectlanguage{greek}'' \cite{[JAVA2]}, με αποτέλεσμα να μπορεί να τρέξει σε όλες τις πλατφόρμες που υποστηρίζουν \selectlanguage{english}Java\selectlanguage{greek} χωρις να χρείαζεται επαναμεταγλώττιση \cite{[JAVA3]}. Από το 2016 η \selectlanguage{english}Java\selectlanguage{greek} έχει ανέλθει σε μία από τις δημοφιλέστερες εν ενεργεία γλώσσες προγραμματισμού \cite{[JAVA4], [JAVA5], [JAVA6]}.

Η \selectlanguage{english}Java\selectlanguage{greek} εμφανίστηκε στο προσκήνιο το 1991, χάρις στον \selectlanguage{english}J. Gosling\selectlanguage{greek} και τους συνεργάτες του. Η πρώτη επίσημη δημοσίευση έγινε το 1996 από την εταιρεία \selectlanguage{english}Sun Microsystems\selectlanguage{greek} \cite{[JAVA8]}. Σύντομα, ενσωματώθηκε στους σημαντικότερους φυλλομετρητές ιστού (\selectlanguage{english}web browsers\selectlanguage{greek}), και σε ιστοσελίδες (\selectlanguage{english}web pages\selectlanguage{greek}), πράγμα που την μετέτρεψε σε ισχυρό προγραμματιστικό εργαλείο. Η \selectlanguage{english}Java\selectlanguage{greek} δέχθηκε αρκετές επιρροές από τη \selectlanguage{english}C++\selectlanguage{greek}, όσον αφορά το συντακτικό κομμάτι. Όντας επίσης μιας αντικειμενοσταφής γλώσσα προγραμματισμού, η \selectlanguage{english}Java\selectlanguage{greek} θεμελιώνεται σε κλάσεις (\selectlanguage{english}classes\selectlanguage{greek}) και κάθε ομάδα δεδομένων αποτελεί ένα ``\textit{αντικείμενο}'' (\selectlanguage{english}\textit{object}\selectlanguage{greek}). Εξαίρεση αποτελούν οι πρωτογενείς τύποι δεδομένων (πχ \selectlanguage{english}\textit{integers, floating points, boolean values}\selectlanguage{greek} κλπ)

To 2005, η \selectlanguage{english}Google\selectlanguage{greek} ανακοίνωσε ένα νέο λειτουργικό σύστημα προοριζόμενο για φορητές συσκευές με το όνομα \selectlanguage{english}Android\selectlanguage{greek}. Πρόκειται για μια παραλλαγή του πυρήνα του λειτουργικού συστήματος \selectlanguage{english}Linux\selectlanguage{greek}, με προσθήκες από κάποιες ακόμη βιβλιοθήκες ανοιχτού λογισμικού. Το λογισμικό \selectlanguage{english}Android\selectlanguage{greek} υποστηρίζεται από συσκευές 3ης γενιάς (\selectlanguage{english}smartphones, tablets\selectlanguage{greek}) καθώς και άλλα προϊόντα της \selectlanguage{english}Google\selectlanguage{greek} (πχ \selectlanguage{english}\textit{Android Auto, Android TV, Wear OS}\selectlanguage{greek} κλπ). οι εφαρμογές που αναπτύσσονται με αυτό το λειτουργικό σύστημα υλοποιούνται με τη χρήση της πλατφόρμας \selectlanguage{english}\textit{Android Studio SDK} \cite{[JAVA9]}\selectlanguage{greek} και της γλώσσας \selectlanguage{english}Java \cite{[JAVA10]}\selectlanguage{greek}. Την πλατφόρμα αυτή συνοδεύουν και άλλα χρήσιμα εργαλεία, όπως αποσφαλματωτής (\selectlanguage{english}debugger\selectlanguage{greek}), βιβλιοθήκες λογισμικού και προσομοιωτής (\selectlanguage{english}emulator\selectlanguage{greek}) \cite{[JAVA11]}.

\selectlanguage{english}
\begin{lstlisting}[language=Java, caption=\selectlanguage{greek}Παράδειγμα κώδικα σε \selectlanguage{english}Java]
// This is an example of a single line comment using two slashes

/* This is an example of a multiple line comment using the slash and asterisk.
 This type of comment can be used to hold a lot of information or deactivate
 code, but it is very important to remember to close the comment. */

package fibsandlies;
import java.util.HashMap;

/**
 * This is an example of a Javadoc comment; Javadoc can compile documentation
 * from this text. Javadoc comments must immediately precede the class, method, or field being documented.
 */
public class FibCalculator extends Fibonacci implements Calculator {

    private static Map<Integer, Integer> memoized = new HashMap<Integer, Integer>();

    /*
     * The main method written as follows is used by the JVM as a starting point for the program.
     */
    public static void main(String[] args) {
        memoized.put(1, 1);
        memoized.put(2, 1);
        System.out.println(fibonacci(12)); //Get the 12th Fibonacci number and print to console
    }

    /**
     * An example of a method written in Java, wrapped in a class.
     * Given a non-negative number FIBINDEX, returns
     * the Nth Fibonacci number, where N equals FIBINDEX.
     * @param fibIndex The index of the Fibonacci number
     * @return The Fibonacci number
     */
    public static int fibonacci(int fibIndex) {
        if (memoized.containsKey(fibIndex)) {
            return memoized.get(fibIndex);
        } else {
            int answer = fibonacci(fibIndex - 1) + fibonacci(fibIndex - 2);
            memoized.put(fibIndex, answer);
            return answer;
        }
    }
}


\end{lstlisting}



\subsection{\selectlanguage{english}JavaScript}
\selectlanguage{greek}
Μιλαμε εισαγωγικα για τη γλωσσα αυτη, ιδεες απο τη διπλωματικη του παιδιου και περναμε στην Ρεακτ. Μιλαμε για τα γνωρισματα και τα πλεονεκτηματα της (παρε την περιληψη που εγραψες).

\subsubsection{\selectlanguage{english}React Native}


\chapter{Ανάλυση Απαιτήσεων Συστήματος}
\label{chap3}

Η εφαρμογή που πραγματεύεται η παρούσα διπλωματική εργασία προορίζεται για χρήση πάνω σε πλατφόρμες κινητών συσκευών που είναι συνδεδεμένες σε δίκτυο. Σκοπός της εφαρμογής είναι να βελτιστοποιήσει τη διαδικασία της ενημέρωσης για κοινωνικοπολιτισμικά δρώμενα και να προσφέρει μια ολοκληρωμένη εμπειρία στους χρήστες της. Αυτό σημαίνει ότι ο χρήστης θα μπορεί να δημιουργήσει τον προσωπικό του λογαριασμό και να ενημερώνεται για εκδηλώσεις που λαμβάνουν χώρα αυτή τη στιγμή γύρω του, όποτε το επιθυμεί. Ο χρήστης θα μπορεί να περιηγηθεί στην κεντρική διεπιφάνεια της εφαρμογής με τη βοήθεια μιας διεπαφής χάρτη, να αναρτήσει σχόλια σχετικά με κάποιο γεγονός στην τοποθεσία του και να αλληλεπιδράσει με άλλους χρήστες που βρισκονται στην περιοχή. Η εφαρμογή θα βοηθάει στο συντονισμό όλων των ατόμων που σχετίζονται με την εκδήλωση, είτε αυτοί είναι οι διοργανωτές, είτε είναι οι συμμετέχοντες.

Το παρόν κεφάλαιο καταπιάνεται με την ανάλυση των απαιτήσεων του συστήματος. Θα γίνει μια εκτενής επεξήγηση κάθε λειτουργικότητας της εφαρμογής και θα παρουσιαστούν οι βασικές αρχές που τις διέπουν.

\section{Γενική Περιγραφή}
Η εφαρμογή αποτελείται από τρία μέρη: (α) τον \tl{client}, (β) τον \tl{server} και (γ) τη βάση δεδομέων. Ο \tl{client} είναι η διεπιφάνεια όπου δρα ο χρήστης, ενώ ο \tl{server} αποτελείται από το σύνολο των μεθόδων υπεύθυνων για την εκτέλεση αιτημάτων από τον \tl{client} (βλ. Σχ. \ref{appecosystem}).

Ο \tl{client} αποτελείται από ένα δρομολογητή (\tl{router}) που συνίσταται από τις διεπιφάνειες της εφαρμογής με τις οποίες μπορεί να αλληλεπιδράσει ο χρήστης. Η κεντρική διεπιφάνεια αποτελείται από μια διεπαφή χάρτη (\tl{Apple Maps API}) για την προβολή σημείων ενδιαφέροντος, τα οποία από εδώ και στο εξής θα αποκαλούνται \tl{\textit{hotspots}}. Ο \tl{router} απαρτίζεται ακόμη από τις σελίδες \tl{\textit{comments, hotspots, profile, new hotspot}}. Καθεμία από αυτες θα αναλυθεί σε επόμενη ενότητα (βλ. ΠΟΙΑ ενότητα). Η υλοποίηση και ο έλεγχος του \tl{client} έγιναν με τη βοήθεια φυσικής συσκευής με λειτουργικό \tl{iOS} 12, καθώς και ειδικού \tl{SDK} για αυτό το σκοπό (περισσότερες λεπτομέρειες θα δωθούν στο Κεφ. \ref{chap6}).

Ο \tl{server} καθορίζει τις λειτουργίες προς εκτέλεση που αντιστοιχούν στις αιτήσεις που δημιουργεί ο χρήστης μέσω του \tl{UI} της εφαρμογής. Οι λειτουργίες αυτές συνοπτικά είναι η αποθήκευση των διαπιστευτηρίων και των προσωπικών πληροφοριών του χρήστη, η δημιουργία λογαριασμού (\tl{profile}), η δημιουργία και επεξεργασία δημοσιεύσεων (\tl{create/update hotspot}), η δημιουργία σχολίων (\tl{create comment}), η λήψη και αποθήκευση φωτογραφιών κλπ. Η κατασκευή του \tl{server} έχει δομή τέτοια ώστε να αποσκοπεί στη συνεργατική δράση του με τον \tl{client}, με αποδοτικό και αποτελεσματικό τρόπο.

Η εφαρμογή θα διαθέτει επίσης μια βάση δεδομέων για την αποθήκευση των χρηστών, των δημοσιεύσεων, των σχολίων και άλλων στοιχείων που αφορούν το χρήστη, είτε άμεσα (στοιχεία λογαριασμού, δημοσιεύσεις χρήστη, σχολιασμοί χρήστη κλπ), είτε έμεσσα (σχόλια σε αναρτήσεις του χρήστη, σχόλια σε δημοσιεύσεις άλλων χρηστών κλπ.).  


\begin{figure}[h]
    \centering
    \includegraphics[scale=1]{figures/app-ecosystem.png}
    \caption{Περιβάλλον συστήματος εφαρμογής}
    \label{appecosystem}
\end{figure}

\section{Απαιτήσεις Συστήματος}
Σε αυτή την ενότητα αναφέρονται συνοπτικά οι απαιτήσεις του συστήματος που εξασφαλίζουν την ολοκληρωμένη λειτουργία της εφαρμογής. Πρόκειται για τις λειτουργικότητες της εφαρμογής οι οποίες προσφέρουν την επιθυμητή εμπειρία στο χρήστη. Στο σύνολό τους, οι λειτουργικότητες θα πρέπει να εξυπηρετούν το σκοπό για τον οποίο αναπτύχθηκε η εφαρμογή (βλ. ενότητα 3.2.1). Επίσης, προκειμένου να εξασφαλιστεί η ολοκληρωμένη και καλή εμπειρία του χρήστη, είναι απαραίτητο να ληφθούν υπόψη και κάποιες μη λειτουργικές απαιτήσεις. οι μη λειτουργικές απαιτήσεις δεν έχουν να κάνουν με την υλοποίηση της διεπιφάνειας του χρήστη, αλλά με τη γενικότερη λειτουργικότητα της εφαρμογής. Η ανάλυση αυτών γίνεται στην ενότητα 3.2.2.

\subsection{Λειτουργικές Απαιτήσεις Συστήματος}
Η διεπιφάνεια του χρήστη θα πρέπει να φέρει τις παρακάτω λειτουργικότητες:

\begin{enumerate}
    \item \textbf{Εγγραφή χρήστη} \\
    Ο χρήστης μπορεί να πραγματοποιήσει εγγραφή στην εφαρμογή, με τη συμπλήρωση κατάλληλης φόρμας.
    \item \textbf{Σύνδεση χρήστη} \\
    Υπάρχοντες χρήστες μπορούν να πραγματοποιήσει σύνδεση στην εφαρμογή, μέσω της συμπλήρωσης κατάλληλης φόρμας με τα διαπιστευτήριά του.
    \item \textbf{Ταυτοποίηση χρήστη} \\
    Η ταυτότητα του χρήστη πιστοποιείται μέσω της έκδοσης \tl{token} με το \tl{OAuth} πρωτόκολλο.
    \item \textbf{Δημοσίευση κατάστασης -- \tl{\textit{hotspot}}} \\
    Ο χρήστης μπορεί να αναρτήσει μια δημοσίευση, σχετική με κάποιο εγγύς γεγονός ή εκδήλωση. Η ανάρτηση θα εμφανίζεται στην διεπιφάνεια χάρτη της κεντρικής σελίδας, στην τοποθεσία του χρήστη. Κατα τη δημιουργία \tl{hotspot} είναι εφικτές οι ακόλουθες επιλογές:
    \begin{itemize}
        \item προσθήκη περιγραφής (απαιτείται)
        \item προσθήκη φωτογραγίας (προεραιτικά)
        \item προσδιορισμός διάρκειας ισχύος της δημοσίευσης (απαιτείται)
    \end{itemize}
    \item \textbf{Ανάγνωση \tl{hotspot} άλλων χρηστών} \\
    Ο χρήστης μπορεί να «ανοίξει» και να διαβάσει \tl{hotspots} άλλων χρηστών στην περιοχή του ή σε άλλη περιοχή, μετακινόντας το χάρτη.
    \item \textbf{Δημιουργία Σχολίων} \\
    Ο χρήστης μπορεί να σχολιάσει σε όλα τα \tl{hotspots} (δικά του και άλλων χρηστών).
    \item \textbf{Προβολή προσωπικών \tl{hotspots}} \\
    Ο χρήστης μπορεί να επισκεφτεί τη λίστα με όλα τα προσωπικά του \tl{hotspots}. Η λίστα είναι ταξινομημένη με το πιο πρόσφατο \tl{hotspot} να φαίνεται πρώτο και το παλαιότερο \tl{hotspot} να φαίνεται τελευταίο.
    \item \textbf{Προβολή πρόσθετων πληροφοριών \tl{hotspot}} \\
    Ο χρήστης μπορεί να επιβλέπει την κατάσταση όλων των \tl{hotspots} (δικών του και άλλων χρηστών) μέσω δεικτών μέτρησης \tl{views} και \tl{comments}. Ο δείκτης \tl{views} είναι υπεύθυνος για τον αριθμό νέων αναγνώσεων ενός \tl{hotspot}, ενώ ο δείκτης \tl{comments} δείχνει τον αριθμό των σχολίων ενός \tl{hotspot}. 
    \item \textbf{Επιλογή φίλτρου αναζήτησης} \\
    Ο χρήστης μπορεί να επιλέξει μια κατηγορία ενδιαφέροντος και να πραγματοποιήσει αναζήτηση στον χάρτη σημείων που ανήκουν σε αυτή. Με το σύρσιμο της οθόνης νέα σημεία εμφανίζονται στο χάρτη.
    \item \textbf{Αναίρεση φίλτρου αναζήτησης} \\
    Ο χρήστης μπορεί να καταργήσει τα επιλεγμένο φίλτρο για να επιλέξει νέο, ή να «καθαρίσει» το χάρτη.
    \item \textbf{Μπάρα Αναζήτησης} \\
    Ο χρήστης μπορεί να πληκτρολογήσει κάποιο σημείο ενδιαφέροντος (\tl{point of interest}) και να πραγματοποιήσει συγκεκριμένη αναζήτηση του επιλεγμένου σημείου. 
    \item \textbf{Άνοιγμα κεντρικού μενού πλοήγησης} \\
    Ο χρήστης μπορεί να πατήσει στο κεντρικό μενού (\tl{\textit{FAB}}) και να επιλέξει τη σελίδα στην οποία θέλει να μεταβεί. Ο χρήστης μπορεί να πλοηγηθεί στις παρακάτω σελίδες:
    \begin{itemize}
        \item λίστα προσωπικών \tl{hotspots}
        \item φόρμα δημοσίευσης νέου \tl{hotspot}
        \item προσωπικό λογαριασμό (\tl{profile}) χρήστη
    \end{itemize}
    \item \textbf{Προβολή προσωπικών στοιχείων} \\
    Ο χρήστης μπορεί να μεταβεί στο \tl{profile} του και να δει τα προσωπικά του στοιχεία.
    \item \textbf{Επεξεργασία προσωπικών στοιχείων} \\
    Ο χρήστης μπορεί να επεξεργαστεί τα προσωπικά του στοιχεία και να ανανεώσει το προσωπικό του λογαριασμό μέσω κατάλληλης φόρμας. Μπορεί να ανανεώσει τα διαπιστευτήριά του, την εικόνα \tl{profile}, το όνομά του, τον τόπο διαμονής, την ημερομηνία γέννησης και το γένος.
    \item \textbf{Προβολή στατιστικών μετρήσεων} \\
    Ο χρήστης μπορεί να δει στατιστικά δεδομένα σχετικά με την δράση του στην εφαρμογή, όπως αριθμό \tl{hotspots}, αριθμό σχολίων, αριθμό ατόμων που επισκέφτηκαν τα \tl{hotspots} του χρήστη, ποσοστό αγοριών/κοριτσιών, δημοτικότητα κλπ. 
    \item \textbf{Αποσύνδεση} \\
    Ο χρήστης μπορεί να αποσυνδεθεί από το λογαριασμό του κάνοντας \tl{logout} (από τη σελίδα του \tl{profile} του).
\end{enumerate}



\subsection{Μη Λειτουργικές Απαιτήσεις Συστήματος}
Η εφαρμογή θα πληροί και ορισμένες μη λειτουργικές απαιτήσεις για μια ολοκληρωμένη εμπειρία χρήστη. Οι απαιτήσεις αυτές απαριθμούνται στη συνέχεια.

\begin{enumerate}
    \item \textbf{Απόδοση -- \tl{Performance}} \\
    Ο χρόνος απόκρισης του συστήματος σην εξυπηρέτηση αιτημάτων από το \tl{server} θα πρέπει να είναι μικρός. Το σύστημα θα πρέπει να μπορεί να ανταποκρίθεί σε μεγάλο αριθμό αιτημάτων.
    \item \textbf{Ταχύτητα ανάκτησης -- \tl{Recoverability}} \\
    Το σύστημα θα πρέπει να μπορεί να ανακτήσει την επιθυμητή κατάσταση λειτουργίας γρήγορα, σε περίπτωση αποτυχίας ή απότομου τερματισμού.
    \item \textbf{Υποστήριξη -- \tl{Documentation}} \\
    Η εφαρμογή θα πρέπει να συνοδεύεται από έγγραφη τεκμηρίωση. Η τεκμιρίωση θα πρέπει να είναι επεξηγηματική και εκτενής προκειμένου να καλύπτει όλα τα πιθανά ζητήματα που μπορεί να προκύψουν.
    \item \textbf{Ασφάλεια -- \tl{Security}} \\
    Η εφαρμογή θα πρέπει να εξασφαλίζει την ασφάλεια των προσωπικών δεδομένων των χρηστών της. Ευαίσθητα δεδομένα όπως κωδικοί ασφαλείας θα πρέπει να κρυπτογραφούνται πριν την αποθήκευσή τους στη βάση, για την αποφυγή κατάχρησης από κακόβουλους χρήστες.
    \item \textbf{Επεκτασιμότητα -- \tl{Extensibility/Scalability}} \\
    Η υλοποίηση της εφαρμογής θα πρέπει να στηρίζεται σε τεχνολογίες που επιτρέπουν την μελλοντική επέκτασή της με νέα χαρακτηριστικά και πρόσθετες εφαρμογές. Επιπλέον, η εφαρμογή θα πρέπει να μπορεί να διαχειριστεί μεγάλο αριθμό χρηστών.
\end{enumerate}



\section{Αρχιτεκτονική Εφαρμογής}
%Σε 2-3 παραγράφους εξηγούμε ότι θα ακολουθήσει η ανάλυση του προβλήματος που πραγματεύεται η διπλωματική.
Όπως είδαμε στο  σχ. \ref{appecosystem}, το περιβάλλον της εφαρμογής διαφοργώνεται από τρεις βασικούς συντελεστές: τον \tl{client}, τον \tl{server} και τη βάση δεδομένων. Ο \tl{client} θα πρέπει να είναι συνδεδεμένος σε δίκτυο προκειμένου να εδραιωθεί η επικοινωνία με τον \tl{server}. Ο \tl{server} της εφαρμογής φιλοξενείται από έναν κεντρικό εξυπηρετητή μέσω μιας διαδικασίας γνωστής ως \tl{hosting}. [υλοποιώ και επανέρχομαι να πω περισσότερα]. Η διεπικοινωνία μεταξύ \tl{client-server} είναι υπεύθυνη για την σωστή λειτουγία της εφαρμογής και την διασφάλιση μιας ολοκληρωμένμης εμπειρίας από την πλευρά του χρήστη. 

Ο \tl{client} είναι υπεύθυνος για την διαμόρφωση ενός διδραστικού περιβάλλοντος για τον χρήστη. Το περιβάλλον αυτό αποτελείται από έναν \tl{router} ο οποίος είναι υπεύθυνος για την πλοήγηση του χρήστη στις διάφορες σελίδες της εφαρμογής. Σε κάθε σελίδα, υπάρχουν διαδραστικά στοιχεία όπως γραφικές διεπιφάνειες, διεπαφές, πλήκτρα ενεργειών και στοιχεία εισόδου και διάφορα άλλα στοιχεία αλληλεπίδρασης, όλα με σκοπό την παροχή μιας πλήρους εμπειρίας στο χρήστη.

Ο \tl{server} αναλαμβάνει να εξυπηρετήσει τα αιτήματα που δημιουργεί ο χρήστης από την πλευρά του \tl{client}. Κάθε φορά που ο χρήστης πραγματοποιεί μια ενέργεια που απαιτεί δεδομένα από τον \tl{server}, τότε δημιουργείται ένα αίτημα το οποίο καλείται νβα εξυπηρετήσει ο \tl{server}. Η προβολή κάποιου \tl{hotspot}, η δημιουργία ενός σχολίου, η επεξεργασία του λογαριασμού του χρήστη, αποτελούν μερικά παραδείγματα τέτοιων αιτημάτων.

Η βάση δεδομένων αποτελεί τον αποθηκευτικό χώρο της εφαρμογής. Όλα τα δεδομένα, από το λογαριασμό του χρήστη μέχρι και τον αριθμό προβολών ενός \tl{hotspot}, αποθηκεύονται σε συλλογές εντός της βάσης. Έτσι, κάθε φορά που ζητούνται δεδομένα, ο \tl{server} πραγματοποιεί αναζήτηση κατάλληλης μορφής στη βάση και επιστρέφει τα απαραίτητα δεδομένα στον \tl{client}.


\subsection{Υποδομή \tl{Client} και Σενάρια Χρήσης (\textit{\tl{Frontend}})}
Στις επόμενες ενότητες ακολοθεί μια επεξηγηματική ανάλυση της αρχιτεκτονικής του παραπάνω συστήματος. Θα παρουσιαστούν οι υποδομές της εφαρμογής και οι οθόνες που τις θεμελιώνουν. Έπειτα, θα αναλυθούν όλα τα πιθανά σενάρια χρήσης (\tl{use cases}) της εφαρμογής. 

\subsubsection{Οθόνες Εφαρμογής}
Η εφαρμογή δομείται σε τέσσερα κύρια μέρη: ταυτοποίηση, κεντρική σελίδα χάρτη, \tl{hotspots} και \tl{profile}. Οι αντίστοιχες οθόνες (\tl{screens}) αυτών είναι:
\item (α) οθόνη έναρξης (\textit{\tl{WelcomeScreen}})

Εκεί μεταφέρεται ο χρήστης την πρώτη φορά που ανοίγει την εφαρμογή, ή όταν αποσυνδεθεί από το λογαριασμό του.

\item (β) οθόνη ταυτοποίσης χρήστη (\textit{\tl{SignUpScreen \& LoginScreen}}) 

Η οθόνη ταυτοποίησης χρήστη αποτελείται από δυο \tl{tabs} (\tl{Register, Login}), καθένας εκ των οποίων περιέχει φόρμα κατάλληλης μορφής την οποία καλείται να συμπληρώσει ο χρήστης.

\item (γ) κεντρική οθόνη χάρτη (\textit{\tl{HomeScreen}})

Η κεντρική οθόνη αποτελείται κυρίως από τη διεπαφή χάρτη (\tl{Apple Map API}). Στον χάρτη φαίνονται όλα τα \tl{hotspots} και ο χρήστης μπορεί να ανανεώνει το περιεχόμενο, αλλάζοντας τη θέση του χάρτη.

\item (ε) οθόνη δημιουργίας νέου \tl{hotspot} (\textit{\tl{CreateHotspotScreen}})

Η οθόνη αυτή συνίσταται από μια φόρμα όπου ο χρήστης εισάγει τα στοιχεία του \tl{hotspot} προς δημιουργία.

\item (δ) οθόνη επεξεργασίας \tl{hotspot} (\textit{\tl{EditHotspotScreen}})

Εάν ο χρήστης το επιθυμεί, μπορεί να επεξεργαστεί ένα από τα δικά του \tl{hotspots} εδώ.

\item (στ) οθόνη προσωπικών \tl{hotspot} (\textit{\tl{HotspotListScreen}})

Εδώ μπορεί να πλοηγηθεί ο χρήστης όταν επιθυμεί να προβάλλει όλα τα \tl{hotspots} που έχει δημιουργήσει ο ίδιος.  

\item (ζ) οθόνη προσωπικού \tl{profile} (\textit{\tl{ProfileScreen}})

Σε αυτή την οθόνη, ο χρήστης μπορεί να προβάλλει τα στοιχεία του λογαριασμού του.

\item (η) οθόνη επεξεργασίας \tl{profile} (\textit{\tl{EditProfileScreen}})

Εάν ο χρήστης επιθυμεί να αλλάξει τα στοιχεία του προσωπικού του λογαριασμού ή να επεξεργαστεί κάποια από αυτά, μπορεί να το κάνει σε αυτή τη σελίδα. Η οθόνη αποτελείται από μια φόρμα η οποία περιέχει τα αρχικά στοιχεία του χρήστη πριν από την αλλαγή. Για την ολοκλήρωση της διαδικασίας απαιτείται ο κωδικός του χρήστη για επιβεβαίωση. 

\item (θ) οθόνη στατιστικών δεδομένων (\textit{\tl{StatisticsScreen}})

Στατιστικά δεδομένα σχετικά με τη χρήση της εφαρμογής και την αλληλεπίδραση με άλλους χρήστες εμφανίζονται στην οθόνη στατιστικών δεδομένων.


\subsubsection{Σενάρια Χρήσης Εφαρμογής}

\paragraph{Εγγραφή Χρήστη στην Εφαρμογή (\textit{\tl{Register}})}
\begin{enumerate}
    \item Ο χρήστης ανοίγει την εφαρμογή και πατάει το πλήκτρο ``\textit{\tl{Get Started}}'' στην οθόνη έναρξης (\textit{\tl{WelcomeScreen}}).
    \item Ο χρήστης συμπλήρώνει τη φόρμα εγγραφής που βρίσκεται στο \tl{tab} με την ονομασία \tl{register}. Στη φόρμα ο χρήστης πρέπει να εισάγει όλες τις απαιτούμενες πληροφορίες. Τα απαιτούμενα πεδία είναι τα εξής:
    \begin{itemize}
        \item πλήρες όνομα (\textit{\tl{fullname}})
        \item όνομα χρήστη (\textit{\tl{username}})
        \item ηλεκτρονική διεύθυνση (\textit{\tl{email}})
        \item κωδικός (\textit{\tl{password}})
        \item επαλήθευση κωδικού (\textit{\tl{confirm password}})
        \item ημερομηνία γέννησης (\textit{\tl{birthdate}})
        \item πόλη (\textit{\tl{city}})
        \item γένος χρήστη (\textit{\tl{gender}})
    \end{itemize}
    Εάν ο χρήστης το επιθυμεί, μπορεί να επιλέξει μια φωτογραφία \tl{profile} από τη συλλογή φωτογραφιών του ή να βγάλει καινούρια μέσω της κάμερας της συσκευής.
    \item Στην περίπτωση που δεν υπάρχουν σφάλματα κατά την υποβολή της φόρμας εγγραφής, ο χρήστης εγγράφεται στην εφαρμογή με επιτυχία. Η εφαρμογή τον πηγαίνει αυτόματα στην κεντρική σελίδα της διεπαφής χάρτη (\textit{\tl{HomeScreen}}).
\end{enumerate}

\paragraph{Σύνδεση Χρήστη στην Εφαρμογή (\textit{\tl{Login}})}
\begin{enumerate}
    \item Ο χρήστης ανοίγει την εφαρμογή και πατάει το πλήκτρο ``\textit{\tl{Get Started}}'' στην οθόνη έναρξης (\textit{\tl{WelcomeScreen}}).
    \item Ο χρήστης συμπλήρώνει τη φόρμα σύνδεσης που βρίσκεται στο \tl{tab} με την ονομασία \tl{login}. Στη φόρμα ο χρήστης πρέπει να εισάγει όλες τις απαιτούμενες πληροφορίες. Τα απαιτούμενα πεδία είναι τα εξής:
    \begin{itemize}
        \item ηλεκτρονική διεύθυνση (\textit{\tl{email}})
        \item κωδικός πρόσβασης (\textit{\tl{password}})
    \end{itemize}
    \item Στην περίπτωση που δεν υπάρχουν σφάλματα κατά την υποβολή της φόρμας σύνδεσης, ο χρήστης επαναφέρεται αυτόματα στην αρχική σελίδα (\textit{\tl{HomeScreen}}).
\end{enumerate}

\paragraph{Δημιουργία Νέου \tl{Hotspot}}
\begin{enumerate}
    \item Ο χρήστης πατάει πάνω στο \tl{FAB} στο κάτω δεξία μέρος της οθόνης στην \tl{HomeScreen}.
    \item Ο χρήστης πατάει επάνω στο εικονίδιο με το σύμβολο ``\textit{\tl{plus}}'' από το κεντρικό μενού πλοήγησης που εμφανίζεται.
    \item ο χρήστης μεταφέρεται στη σελίδα δημιουργίας νέου \tl{Hotspot} (\textit{\tl{CreateHotspotScreen}}).
    
    \item Ο χρήστης συμπλήρώνει τη φόρμα δημιουργίας νέου \tl{Hotspot}. Τα απαιτούμενα πεδία είναι τα εξής:
    \begin{itemize}
        \item περιγραφή (\textit{\tl{description}})
        \item χρονική διάρκεια ισχύος σε λεπτά (\textit{\tl{validity}})
        
    \end{itemize}
    Εάν ο χρήστης το επιθυμεί, μπορεί να προσθέσει μια φωτογραφία από τη συλλογή φωτογραφιών του ή να βγάλει καινούρια μέσω της κάμερας της συσκευής.
    \item Στην περίπτωση που δεν υπάρχουν σφάλματα κατά τη δημιουργία νέου \tl{Hotspot}, η διαδικασία ολοκληρώνεται επιτυχώς και ο χρήστης μεταφέρεται αυτόματα στην αρχική σελίδα. Το νέο \tl{Hotspot} θα εμφανίζεται στο χάρτη στην τοποθεσία του χρήστη.
\end{enumerate}

\paragraph{Γρήγορη Προβολή \tl{Hotspot}}
\begin{enumerate}
    \item Στο χάρτη της αρχικής οθόνης, ο χρήστης πατάει πάνω στο εικονίδιο ενός \tl{hotspot} (\textit{\tl{hotspot marker}}).
    \item Οι λεπτομέρειες του \tl{hotspot} εμφανίζονται σε ένα ``\textit{παράθυρο}'' ενσωματωμένο στο χάρτη (\textit{\tl{callout}}). ο χρήστης μπορεί να δει την περιγραφή, τον αριθμό σχολίων και προβολών, και αν υπάρχει, την εικόνα του επιλεγμένου \tl{hotspot}.
\end{enumerate}

\paragraph{Προβολή Λεπτομερειών \tl{Hotspot}}
\begin{enumerate}
    \item Στο χάρτη της αρχικής οθόνης, ο χρήστης πατάει πάνω στο εικονίδιο ενός \tl{hotspot} (\textit{\tl{hotspot marker}}).
    \item Οι λεπτομέρειες του \tl{hotspot} εμφανίζονται σε ένα ``\textit{παράθυρο}'' ενσωματωμένο στο χάρτη (\textit{\tl{callout}}). ο χρήστης μπορεί να δει την περιγραφή, τον αριθμό σχολίων και προβολών, και αν υπάρχει, την εικόνα του επιλεγμένου \tl{hotspot}.
    \item Ο χρήστης πατάει πάνω στο εικονίδιο ``\textit{\tl{comments}}'' και μεταφέρεται αυτόματα στην οθόνη λεπτομερειών του συγκεκριμένου \tl{hotspot}.
\end{enumerate}

\paragraph{Δημιουργία Σχολίου}
\begin{enumerate}
    \item Ο χρήστης πηγαίνει στη σελίδα λεπτομερειών ενός \tl{hotspot} (\textit{\tl{CommentsScreen}}) ακολουθώντας τη διαδικασία που περιγράφεται στην παράγραφο \textit{Προβολή Λεπτομερειών \tl{Hotspot}}.
    \item Ο χρήστης πλοηγείται στο κάτω μέρος της σελίδας.
    
    Αλλιώς,
    \item Ο χρήστης σύρει προς τα αριστερά ένα από τα υπάρχοντα σχόλια και πατάει το εικονίδιο ``\textit{\tl{reply}}''.
    \item Ο χρήστης εισάγει το σχόλιό του στο \textit{\tl{comment box}} με επιγραφή ``\textit{\tl{Add a comment...}}''.
    \item Μόλις ο χρήστης ολοκληρώσει το σχόλιό του, πατάει στο εικονίδιο ``\textit{\tl{send}}''.
    \item Σε περίπτωση που το σχόλιο δεν έχει σφάλματα, η διαδικασία ολοκληρώνεται επιτυχώς και το νέο σχόλιο εμφανίζεται τελευταίο στο τμήμα σχολίων της οθόνης.
\end{enumerate}

\paragraph{Προβολή Περισσότερων Σχολίων}
\begin{enumerate}
    \item Ο χρήστης πηγαίνει στη σελίδα λεπτομερειών ενός \tl{hotspot} (\textit{\tl{CommentsScreen}}) ακολουθώντας τη διαδικασία που περιγράφεται στην παράγραφο \textit{Προβολή Λεπτομερειών \tl{Hotspot}}.
    \item Ο χρήστης πλοηγείται στο κάτω μέρος της σελίδας.
    \item Ο χρήστης πατάει το πλήκτο με επιγραφή ``\tl{\textit{more}}''. Πέντε νέα σχόλια εμφανίζονται στο τμήμα σχολίων της οθόνης.
\end{enumerate}

\paragraph{Συγκεκριμένη Αναζήτηση Σημείου Ενδιαφέροντος}
\begin{enumerate}
    \item Στη σελίδα χάρτη (\textit{\tl{HomeScreen}}) ο χρήστης πατάει εντός της μπάρας αναζήτησης με επιγραφή ``\tl{\textit{Search a hotspot...}}'' που βρίσκεται στο πάνω μέρος της οθόνης.
    \item Ο χρήστης πληκτρολογεί το όνομα ή κάποια άλλη λέξι-κλειδί που σχετίζεται με το σημείο ενδιαφέροντος.
    \item Ο χρήστης πατάει σε μία από τις προτεινόμενες επιλογές από τη λίστα που εμφανίζεται.
    \item Το επιλεγμένο σημείο ενδιαφέροντος εμφανίζεται στο χάρτη.
\end{enumerate}

\paragraph{Γενικευμένη Αναζήτηση Σημείου Ενδιαφέροντος}
\begin{enumerate}
    \item Στη σελίδα χάρτη (\textit{\tl{HomeScreen}}) ο χρήστης πατάει εντός της μπάρας αναζήτησης με επιγραφή ``\tl{\textit{Search a hotspot...}}'' που βρίσκεται στο πάνω μέρος της οθόνης.
    \item Ο χρήστης πληκτρολογεί το όνομα ή κάποια άλλη λέξι-κλειδί που σχετίζεται με το σημείο ενδιαφέροντος.
    \item Ο χρήστης πατάει το εικονίδιο ``\tl{\textit{search}}''.
    \item Τα αποτελέσματα εμφανίζονται στο χάρτη.
\end{enumerate}

\paragraph{Αναζήτηση Σημείου Ενδιαφέροντος με Φίλτρο}
\begin{enumerate}
    \item Στη σελίδα χάρτη (\textit{\tl{HomeScreen}}) ο χρήστης πατάει το εικονίδιο ``\tl{\textit{menu}}'' που βρίσκεται στο πάνω αριστερά μέρος της οθόνης.
    \item Στο ``\tl{\textit{drawer menu}}'' που εμφανίζεται, ο χρήστης επιλέγει μία από τις κατηγορίες πατώντας επάνω σε αυτή. Οι διαθέσιμες κατηγορίες είναι:
    \begin{itemize}
        \item \tl{\textit{Coffee}}
        \item \tl{\textit{Food}}
        \item \tl{\textit{Drinks}}
        \item \tl{\textit{Sights}}
        \item \tl{\textit{Arts}}
    \end{itemize}
    \item Τα αποτελέσματα εμφανίζονται στο χάρτη. Κάθε φορά που ο χρήστης αλλάζει τη θέση του χάρτη, νέα αποτελέσματα φορτώνονται στη νέα θέση.
\end{enumerate}

\paragraph{Αναίρεση Αναζήτησης}
\begin{enumerate}
    \item Στη σελίδα χάρτη (\textit{\tl{HomeScreen}}) ο χρήστης πατάει το εικονίδιο ``\tl{\textit{menu}}'' που βρίσκεται στο πάνω αριστερά μέρος της οθόνης.
    \item Στο ``\tl{\textit{drawer menu}}'' που εμφανίζεται, ο χρήστης πατάει την επιλογή ``\tl{\textit{Clear}}''.
    \item Τα αποτελέσματα αφαιρούνται από το χάρτη.
\end{enumerate}

\paragraph{Επαναφορά Θέσης Χάρτη (\textit{\tl{Find My Location}})}
\begin{enumerate}
    \item Στη σελίδα χάρτη (\textit{\tl{HomeScreen}}) ο χρήστης πατάει το εικονίδιο ``\tl{\textit{location}}'' που βρίσκεται στο κάτω αριστερά μέρος της οθόνης.
    \item Η θέση του χάρτη επαναφέρεται στην τοποθεσία του χρήστη.
\end{enumerate}

\paragraph{Προβολή Λίστας Προσωπικών \tl{Hotspots}}
\begin{enumerate}
    \item Ο χρήστης πατάει πάνω στο \tl{FAB} στο κάτω δεξία μέρος της οθόνης στην \tl{HomeScreen}.
    \item Ο χρήστης πατάει επάνω στο εικονίδιο με το σύμβολο ``\textit{\tl{hotspots}}'' από το κεντρικό μενού πλοήγησης που εμφανίζεται.
    \item ο χρήστης μεταφέρεται στη σελίδα προσωπικών \tl{Hotspots} (\textit{\tl{HotspotListScreen}}). Στη λίστα εμφανίζονται λεπτομέρειες των \tl{hotspots} του χρήστη. Ο χρήστης μπορεί να δει την περιγραφή, τον αριθμό σχολίων και προβολών του \tl{hotspot}.
\end{enumerate}

\paragraph{Επεξεργασία Προσωπικού \tl{Hotspot}}
\begin{enumerate}
    \item ο χρήστης μεταφέρεται στη σελίδα προσωπικών \tl{Hotspots} (\textit{\tl{HotspotListScreen}}) ακολουθώντας τη διαδικασία που περιγράφεται στην παράγραφο \textit{Προβολή Λίστας Προσωπικών \tl{Hotspots}}.
    \item Ο χρήστης σύρει προς τα δεξιά το \tl{hotspot} που θέλει να επεξεργαστεί. Στη συνέχεια πατάει στο εικονίδιο ``\textit{\tl{edit}}'' που εμφανίζεται εντός του κίτρινου πλαισίου.
    \item Ο χρήστης μεταφέρεται αυτόματα στη σελίδα επεξεργασίας \tl{hotspot} (\textit{\tl{EditHotspotScreen}}).
    \item Ο χρήστης επεξεργάζεται τα επιθυμητά στοιχεία του \tl{hotspot} και όταν τελειώσει πατάει ``\textit{\tl{Save}}'' στο πάνω δεξιά μέρος της οθόνης.
    \item Σε περίπτωση που δεν υπάρχουν σφάλματα, η διαδικασία ολοκληρώνεται επιτυχώς και οι αλλαγές αποθηκεύονται. Το \tl{hotspot} εμφανίζεται πλέον με τις νέες αλλαγές.
    
\end{enumerate}

\paragraph{Διαγραφή Προσωπικού \tl{Hotspot}}
\begin{enumerate}
    \item ο χρήστης μεταφέρεται στη σελίδα προσωπικών \tl{Hotspots} (\textit{\tl{HotspotListScreen}}) ακολουθώντας τη διαδικασία που περιγράφεται στην παράγραφο \textit{Προβολή Λίστας Προσωπικών \tl{Hotspots}}.
    \item Ο χρήστης σύρει προς τα αριστερά το \tl{hotspot} που θέλει να διαγράψει. Στη συνέχεια πατάει στο εικονίδιο ``\textit{\tl{delete}}'' που εμφανίζεται εντός του κόκκινου πλαισίου.
    \item Το \tl{hotspot} διαγάφεται από τη λίστα προσωπικών \tl{hotspots}.
\end{enumerate}

\paragraph{Προβολή Προσωπικού Λογαριασμού (\textit{\tl{Profile}})}
\begin{enumerate}
     \item Ο χρήστης πατάει πάνω στο \tl{FAB} στο κάτω δεξία μέρος της οθόνης στην \tl{HomeScreen}.
    \item Ο χρήστης πατάει επάνω στο εικονίδιο με το σύμβολο ``\textit{\tl{account}}'' (ή την εικόνα \tl{profile}, αν έχει επιλέξει μία) από το κεντρικό μενού πλοήγησης που εμφανίζεται.
    \item ο χρήστης μεταφέρεται στη σελίδα προσωπικού λογαριασμού (\textit{\tl{ProfileScreen}}). Σε αυτή την οθόνη ο χρήστης μπορεί να προβάλλει τα προσωπικά του στοιχεία.
\end{enumerate}

\paragraph{Επεξεργασία Προσωπικού Λογαριασμού (\tl{\textit{Profile}})}
\begin{enumerate}
     \item ο χρήστης μεταφέρεται στη σελίδα προσωπικού λογαριασμού (\textit{\tl{ProfileScreen}}) ακολουθώντας τη διαδικασία που περιγράφεται στην παράγραφο ``\textit{Προβολή Προσωπικού Λογαριασμού}''.
    \item ο χρήστης πατάει στο πλήκτρο ``\textit{\tl{Edit}}'' που βρίσκεται στο επάνω δεξιά μέρος της οθόνης.
    \item ο χρήστης μεταφέρεται στη σελίδα επεξεργασίας του προσωπικού λογαριασμού (\textit{\tl{EditProfileScreen}}). Η οθόνη αυτή αποτελείται από μια φόρμα με τα αρχικά στοιχεία του χρήστη. Ο χρήστης μπορεί να επεξεργαστεί τα προσωπικα του στοιχεία.
    \item Σε περίπτωση επιτυχούς υποβολής της φόρμας επεξεργασίας των στοιχείων του προσωπικού λογαριασμού, τα νέα στοιχεία αποθηκεύονται στο σύστημα και ο χρήστης επαναφέρεται στο \textit{profile} του.
\end{enumerate}

\paragraph{Προβολή Στατιστικών Δεδομένων Λογαριασμού}
\begin{enumerate}
     \item ο χρήστης μεταφέρεται στη σελίδα προσωπικού λογαριασμού (\textit{\tl{ProfileScreen}}) ακολουθώντας τη διαδικασία που περιγράφεται στην παράγραφο ``\textit{Προβολή Προσωπικού Λογαριασμού}''.
    \item ο χρήστης πατάει στο πλήκτρο βέλους με την επιγραφή ``\textit{\tl{Stats for nerds}}'' στο τμήμα ``\textit{\tl{Settings}}''.
    \item Ο χρήστης μεταφέρεται στη σελίδα στατιστικών δεδομένων του προσωπικού του λογαριασμού (\textit{\tl{StatisticsScreen}}). Η οθόνη διαιρείται σε τρία τμήματα: 
    
    (α) \tl{\textit{account stats}}
    
    Στατιστικά δεδομένα που αφορούν τις κινήσεις του χρήστη εντός της εφαρμογής. Τα δεδομένα αυτά είναι ο συνολικός αριθμός των \tl{hotspots} του χρήστη, o συνολικός αριθμός σχολίων του χρήστη και ο συνολικός αιρθμός προβολών των  \tl{hotspots} του χρήστη από άλλους χρήστες.
    
    (β) \tl{\textit{insights}}
    
    Ποσοστιαίες τιμές που αφορούν την δράση και εξέλιξη του χρήστη εντός της εφαρμογής . Οι τιμές αυτές είναι η δημοτικότητα (\tl{popularity}) και η αλληλεπίδραση (\tl{engagement}) του χρήστη.
    
    (γ) \tl{\textit{audience}}
    
    Διάκριση του κοινού του χρήστη με βάση το γένος (θηλυκό ή αρσενικό). 
    
\end{enumerate}





\subsection{Υποδομή \tl{Server} και Βάσης Δεδομένων (\textit{\tl{Backend}})}
Στην προηγούμενη ενότητα, παρουσιάστηκαν οι βασικές λειτουργικότητες που προσφέρει η εφαρμογή στους χρήστες της. Αναλύθηκαν οι υπηρεσίες της εφαρμογής καθώς επίσης και τα σενάρια με τα οποία ο χρήστης μπορεί να αποκτήσει πρόσβαση σε αυτές. Έτσι, όταν ο χρήστης επιθυμεί για παράδειγμα να δημιουργήσει ένα νέο \tl{hotspot}, αρκεί να πλοηγηθεί στην οθόνη δημιουργίας νέου \tl{hotspot}, να συμπληρώσει την αντίστοιχη φόρμα και να την υποβάλλει. Αυτές είναι οι ενέργειες που πρέπει να γίνουν από την πλευρά του \tl{client}. Αυτή η ενότητα ασχολείται με τις ενέργειες που απαιτούνται από την πλευρά του \tl{server} και τον τρόπο με τον οποίο αυτές είναι σχεδιασμένες να αλληλεπιδρούν μεταξύ τους αλλά και με το υπόλοιπο σύστημα.

Είναι προφανής από τα παραπάνω η ανάγκη αποσύμπλεξης των αιτημάτων προς εξυπηρέτηση από το \tl{server}. Κάθε λειτουργία της εφαρμογής χαρακτηρίζεται από ένα σύνολο ενεργειών που πρέπει να εκτελεστούν κάθε φορά προκειμένου να προκληθεί το επιθυμητό αποτέλεσμα. Για παράδειγμα, η δημιουργία ενός \tl{hotspot} πυροδοτεί μια σειρά αιτημάτων προς τον \tl{server} όπως η συσχέτιση του \tl{hotspot} με το δημιουργό του, η αποθήκευση των δεδομένων του \tl{hotspot} στην βάση, η καταγραφή των σχολίων του \tl{hotspot} κλπ. Έτσι, η σχεδίαση θα πρέπει να βρίσκεται σε συμφωνία με τις απαιτήσεις που πρέπει να πληροί το σύστημα. Ανακύπτει λοιπόν το συμπέρασμα πως είναι αδύνατη η σειριακή εξυπηρέτηση όλων των αιτημάτων από τον ίδιο πόρο, καθώς αυτό θα καθιστούσε το σύστημα υπερβολικά αργό και περίπλοκο.

Συνεπώς, η πρακτική που ακολουθεί η δομή του \tl{server} αποσκοπεί στην αποσυσχέτιση διαφορετικών αιτημάτων. Αυτό επιτυγχάνεται με την ιεράρχηση των λειτουργιών που αφορούν τα αιτήματα, με βάση τη φύση τους. Αιτήματα που σχετίζονται με τους χρήστες θα ανατίθενται αποκλειστικά σε εξυπηρετητές ενεργειών των χρηστών, αιτήματα που αφορούν τα \tl{hotspots} θα ανατίθενται αποκλειστικά σε εξυπηρετητές ενεργειών των \tl{hotspots} κοκ. Η αποθήκευση δεδομέων στη βάση βρίσκεται σε συνοχή με την παραπάνω τακτική, αφού τα δεδομένα μοντελοποιούνται σε \tl{collections} βάσει της φύση τους.  Με αυτό τον τρόπο, επιτυγχάνεται η παραλληλοποίηση των λειτουργιών του \tl{server}, η διαίρεση του φορτίου και η διακριτοποίηση των μεθόδων. Το τελευταίο έχει σημαντική συνεισφορά στην διευκόλυνση της σχεδίασης του συστήματος από τον προγραμματιστή, χάρις τον μεγάλο βαθμό επαναληψιμότητας των περισσότερων τμημάτων. 


\subsection{Διεπικοινωνία Μεταξύ Υποδομών}
Το σύστημα \tl{client-server} επικοινωνεί μέσω αιτημάτων συγκεκριμένου σκοπού. Κάθε φορά που ο χρήστης πραγματοποιεί μια ενέργεια εντός της εφαρμογής, ο \tl{client} δημιουργεί ένα αίτημα προς στον \tl{server}. Το αίτημα αυτό δρομολογείται στον κατάλληλο εξυπηρετητή από την πλευρά του \tl{server}. Έπειτα, ακολουθεί η ανάθεση του αιτήματος στον αντίστοιχο πόρο και η εκτέλεση των απαραίτητων λειτουργιών. Το αποτέλεσμα επηρεάζει τα δεδομένα στη βάση και επιστρέφει με κατάλληλη μορφή πίσω στον \tl{client}. Η παραπάνω διαδικασία γίνεται βέλτιστα, μέσω ασύγχρονων μεθόδων διαχείρισης δεδομένων από και προς τη βάση δεδομένων. Η διαδικασία εξυπηρέτησης αιτημάτων είναι παραλληλοποιημένη ώστε πολλά αιτήματα να εξηπηρετούνται από διαφορετικούς πόρους του συστήματος ταυτόχρονα.









%Αυτά που ακολοθούν θα μπουν στο Κεφ. 5 

\subsubsection{\tl{Server Routes}}

Ο \tl{server} χρησιμοποιεί ένα σύστημα δρομολόγησης (\tl{routes}) των αιτημάτων του \tl{client} που προορίζονται προς εξυπηρέτηση. Κάθε ομάδα υπηρεσιών που απευθύνονται σε μια συγκεκριμένη λειτουργία της εφαρμογής αντιστοιχίζεται σε ένα δρομολογητή (\tl{router}). Ο \tl{router} απαρτίζεται από επιμέρους \tl{routes}, καθεμία εκ των οποίων εξυπηρετεί μια συγκεκριμένη λειτουργία. Έτσι, στο παραπάνω παράδειγμα, το αίτημα για τη δημιουργία νέου \tl{hotspot} θα δρομολογηθεί για εξυπηρέτηση από τον \tl{router} που είναι υπεύθυνος για τις ενέργειες΄που αφορούν ένα \tl{hotspot} (\tl{\textit{HotspotRoutes}}). 

Το σύστημα των δρομολογητών έχει σχεδιαστεί με στόχο την απλότητα και την αμεσότητα στην κατανόηση από τον προγραμματιστή (βλ. Σχ. \ref{routes}). Αιτήματα που αφορούν τους χρήστες δρομολογούνται προς εξυπηρέτηση από τον \tl{router} που είναι υπεύθυνος για τις ενέργειες που αφορούν τους χρήστες. Με τον ίδιο τρόπο, Αιτήματα που σχετίζονται με σχόλια χρηστών σε ένα \tl{hotspot}, εξυπηρετούνται από εκείνο τον \tl{router} που είναι υπεύθυνος για τα σχόλια. 

\begin{figure}[h]
    \centering
    \includegraphics[scale=1]{figures/routes.png}
    \caption{Ιεράρχηση του συστήματος δρομολογητών στην εφαρμογή.}
    \label{routes}
\end{figure}

Οι δρομολογητές του συστήματος είναι οι ακόλουθοι:

\begin{itemize}
    \item \tl{\textbf{Hotspot Routes --}} περιλαμβάνει \tl{routes} που αφορούν την εξυπηρέτηση αιτημάτων σχετικά με \tl{hotspots}, όπως δημιουργία, επεξεργασία, διαγραφή, προβολή, ανάκτηση κλπ.
    \item \tl{\textbf{User Routes --}} περιλαμβάνει \tl{routes} που αφορούν την εξυπηρέτηση αιτημάτων σχετικά με χρήστες, όπως εγγραφή, σύνδεση, επεξεργασία, προβολή, ανάκτηση κλπ.
    \item \tl{\textbf{Comment Routes --}} περιλαμβάνει \tl{routes} που αφορούν την εξυπηρέτηση αιτημάτων σχετικά με σχόλια, όπως δημιουργία, προβολή, απαρίθμηση κλπ.
    \item \tl{\textbf{File Routes --}} περιλαμβάνει \tl{routes} που αφορούν την εξυπηρέτηση αιτημάτων σχετικά με συνημμένα αρχεία σε \tl{hotspots}.
    \item \tl{\textbf{Stats Routes --}} περιλαμβάνει \tl{routes} που αφορούν την εξυπηρέτηση αιτημάτων σχετικά με στατιστικά δεδομένα.
\end{itemize}

\subsubsection{\tl{Server Controllers}}

Αφότου τα αιτήματα δρομολογηθούν προς εξυπηρέτηση στους αντίστοιχους \tl{routers}, ακολοθεί η επεξεργασία τους και η εκτέλεσή τους μέσω κατάλληλων μεθόδων. Οι μέθοδοι που αναλαμβάνουν την εκτέλεση των αιτημάτων ονομάζονται ελεγκτές (\tl{controllers}) και αποτελούνται από ένα σύνολο ασύγχρονων εντολών και κλήσεων μεταξύ της βάσης δεδομένων και του \tl{server}.

Όπως και με το σύστημα των δρομολογητών, έτσι και οι ελεγκτές έχουν σχεδιαστεί με στόχο να αποσυμπλέκουν τις ενέργειες που σχετίζονται με κάθε αίτημα (βλ. Σχ. \ref{}). Όλες οι ενέργεεις που αφορούν τα \tl{hotspots} εκτελούνται αποκλειστικά από έναν ελεγκτή. Το ίδιο συμβαίνει και με τις ενέργεεις γύρω από τους χρήστες. Το σύστημα των \tl{controllers} έχει παραπλήσια μορφή με αυτό των \tl{routers} καθώς στόχος της σχεδίασης είναι να υπάρχει ένας βαθμός αναλογίας μεταξύ των διαφόρων τμημάτων,ο οποίος μειώνει αισθητά το χρόνο κατανόησης και βελτιστοποιεί τη διαδικασία υλοποίησης:

\begin{itemize}
    \item \tl{\textbf{Hotspot Controller --}} περιλαμβάνει τις μεθόδους που αφορούν την εξυπηρέτηση αιτημάτων σχετικά με \tl{hotspots}, όπως δημιουργία, επεξεργασία, διαγραφή, προβολή, ανάκτηση κλπ.
    \item \tl{\textbf{User Controller --}} περιλαμβάνει τις μεθόδους που αφορούν την εξυπηρέτηση αιτημάτων σχετικά με χρήστες, όπως εγγραφή, σύνδεση, επεξεργασία, προβολή, ανάκτηση κλπ.
    \item \tl{\textbf{Comment Controller --}} περιλαμβάνει τις μεθόδους που αφορούν την εξυπηρέτηση αιτημάτων σχετικά με σχόλια, όπως δημιουργία, προβολή, απαρίθμηση κλπ.
    \item \tl{\textbf{File Controller --}} περιλαμβάνει τις μεθόδους που αφορούν την εξυπηρέτηση αιτημάτων σχετικά με συνημμένα αρχεία σε \tl{hotspots}.
    \item \tl{\textbf{Stats Controller --}} περιλαμβάνει τις μεθόδους που αφορούν την εξυπηρέτηση αιτημάτων σχετικά με στατιστικά δεδομένα.
    \item \tl{\textbf{Views Controller --}} περιλαμβάνει τις μεθόδους που αφορούν την εξυπηρέτηση αιτημάτων σχετικά με τα \tl{views} ενός \tl{hotspot}.
\end{itemize}








\chapter{Σχεδίαση Συστήματος}
\label{chap4}

Στο προηγούμενο κεφάλαιο έγινε μια εισαγωγική αναφορά στα δομικά μέρη του συστήματος της εφαρμογής, δηλάδή του \tl{frontend} (\tl{client}) και του \tl{backend} (\tl{server} και βάση δεδομένων). Αναπτύχθηκαν οι λειτουργικότητες της εφαρμογής και οι αντίστοιχες οθόνες που θα τις καλύπτουν. Επίσης, έγινε μια αναφορά στα συστατικά του \tl{server} και πως αυτά αλληλεπιδρούν με τη βάση δεδομένων για την λήψη και αποστολή δεδομένων από και προς τον \tl{client}.

Αυτό το κεφάλαιο θα επικεντρωθεί στην σχεδίαση των υποδομών της εφαρμογής. Συγκεκριμένα, θα παρουσιαστούν τα διαγράμματα ροής για τα σενάρια χρήσης της εφαρμογής και θα αναπτυχθούν τα επίμαχα σημεία που χρήζουν της περισσσότερης προσοχής. Έπειτα, θα παρουσιαστούν τα σχέδια των διεπιφανειών που θα συνιστούν κάθε οθόνη της εφαρμογής (βλ. ενότητα 4.1). Στην ενότητα 4.2, θα γίνει αναφορά στην πλευρά του \tl{server}. Θα παρουσιαστεί η διαδικασία σχεδίασης των εξυπηρετητών αιτημάτων και θα μελετηθούν οι τακτικές και οι τεχνολογίες που χρησιμοποιήθηκαν. Τέλος, στην ενότητα 4.3, θα γίνει μια εκτενής ανάλυση του τρόπου με τον οποίο σχεδιάστηκε η βάση δεδομένων για να ανταποκρίνεται βέλτιστα στις απαιτήσεις της εφαρμογής.


\section{Σχεδίαση Μοντέλων των Διεπιφανειών της Εφαρμογής}




\section{Σχεδίαση Συστήματος Εξυπηρέτησης Αιτημάτων}



\section{Σχεδίαση Μοντέλου Βάσης Δεδομένων}




\chapter{Υλοποίηση Συστήματος}
\label{chap5}

Μέχρι στιγμής, έχει γίνει αναφορά στις τεχνολογίες και τις τεχνικές σχεδίασης της εφαρμογής. Το κεφάλαιο αυτό επεκτείνεται στην υλοποίηση της εφαρμογής, αναλύοντας τις μεθόδους που χρησιμοποίηθηκαν για τον προγραμματισμό τόσο του \tl{frontend}, όσο και του \tl{backend}. Η ενότητα 5.1 αφορά το \tl{frontend} κομμάτι, δηλαδή με τον προγραμματισμό των διεπιφανειών χρήστη. Η ενότητα 5.2 καταπιάνεται με την υλοποίηση του \tl{backend}, το οποίο απαρτίζεται από τον εξυπηρετητή αιτημάτων που δημιουργούνται από τις διεπιφάνειες χρήστη και τη βάση δεδομένων στην οποία αποθηκεύονται όλες οι πληροφορίες της εφαρμογής. Στην ενότητα 5.3 παρουσιάζονται τα τεχνολογικά εργαλεία τα οποία χρειάστηκαν στην υλοποίηση. Τέλος, στην ενότητα 5.4 γίνεται μια συνοπτική αναφορά στην πλοτφόρμα προγραμματισμού που επιλέχθηκε για την υλοποίηση της εφαρμογής. 


\section{Λεπτομέρειες Υλοποίησης Διεπιφανειών Χρήστη}
Ιδιαίτερο ενδιαφέρον παρουσιάζουν οι τεχνικές που χρησιμοποιήθηκαν στην υλοποίηση του συστήματος διεπιφανειών χρήστη. Η εφαρμογή αφορά κινητές συσκευές με λειτουργικό \tl{iOS}, επομένως δόθηκε μεγάλη βαρύτητα σε μεθόδους που υιοθετούν οι περισσότερες \tl{native} εφαρμογές της ίδιας κατηγορίας. Κατά τον προγραμματισμό, σχεδόν όλες οι διεπαφές που χρησιμοποιούνται αφορούν αποκλειστικά \tl{iOS} πλατφόρμες προκειμένου να προσφέρουν την καλύτερη δυνατή εμπειρία στο χρήστη.
\newline
\indent
Η εφαρμογή αναφέρεται σε κινητά τελευταίας γενιάς. Αυτό εισάγει κάποιους περιορισμούς στον προγραμματισμό, αλλά έχει ταυτοχρόνως και κάποια πλεονεκτήματα. Για παράδειγμα, οι διαστάσεις της οθόνης στις κινητές συσκευές είναι πολύ μικρότερες από αυτές ενός σταθερού υπολογιστή. Διαδραστικά στοιχεία όπως πλήκτρα, πεδία εισόδου, εικονίδια και άλλα γραφικά θα πρέπει να λαμβάνουν υπόψη τους τον παραπάνω περιορισμό. Συνεπώς, ο προγραμματισμός για \tl{mobile apps} είναι σαφώς πιο δύσκολος από τον προγραμματισμό για \tl{desktop apps}. 
\newline
\indent
Το κυριότερο πλεονέκτημα του προγραμματισμού για να \tl{mobile apps} είναι το γεγονός ότι δεν χρειάζεται να δωθεί ιδιαίτερη βάση σε λεπτομέρειες, αφού οι οθόνες των κινητών δεν επιτρέπουν την σχολαστική ενασχόληση με κάθε στοιχείο ή την υπερβολική λεπτομέρεια των διαδραστικών στοιχείων, αφού αυτό θα καθιστούσε την εμπειρία χρήσης της εφαρμογής κουραστική για το χρήστη. Έτσι, ο προγραμματισμός επικεντρώνεται κυρίως γύρω από το κομμάτι με το οποίο θα αλληλιπιδρά ο χρήστης και τις τεχνικές λεπτομέρειες που συμπεριλαμβάνει. Για να εξασφαλιστεί μια ολοκληρωμένη, αλλά συγχρόνως απλή εμπειρία εντός της εφαρμογής, όλες οι διεπιφάνειες ακολουθούν αυτή την τακτική, όπου το διαδραστικό περιβάλλον αποτελείται αποκλειστικά από διεπαφές που είναι ζωτικές για την σωστή λειτουργία της κάθε οθόνης.


\subsection{Βασικά Προγραμματιστικά Χαρακτηριστικά Οθονών}
Μιλάω για τα τμήματα ζωτικής σημασίας που έχουν οι περισσότερες οθόνες..

\subsubsection{Τεχνική Πλοήγησης στην Εφαρμογή}



\subsubsection{Επικεφαλίδα Οθόνης}



\subsubsection{Κεντρικό Μενού Οθονών}



\subsubsection{Βασικά Πλήκτρα Ενεργειών}



\subsection{Λειτουργικότητα Οθονών Εφαρμογής}



\subsubsection{\tl{WelcomeScreen}}


\subsubsection{\tl{RegisterScreen}}


\subsubsection{\tl{LoginScreen}}


\subsubsection{\tl{HomeScreen}}


\subsubsection{\tl{CreateHotspotScreen}}



\subsubsection{\tl{HotspotListScreen}}


\subsubsection{\tl{CommentScreen}}


\subsubsection{\tl{ProfileScreen}}


\subsection{Τενικές Ενιαίας Διαχείρισης Δεδομένων}



\subsubsection{\tl{React Redux}}



\subsubsection{\tl{Redux Thunk Middleware}}




\subsection{Βιβλιοθήκες και Βοηθητικά Προγραμματιστικά Πακέτα}





\subsection{Ιεράρχηση Αρχείων και Δομή Φακέλων}

\section{Σύστημα αξιολόγησης}

Εδώ περιγράφουμε το σύστημα που χρησιμοποιήσαμε για να αξιολογήσουμε τις τεχνικές μας. Συνήθως, η περιγραφή γίνεται με κείμενο και με ένα block diagram περιγραφής των λειτουργιών του συστήματος.
Αν το σύστημα είναι μεγάλο, τότε συζητήστε με τον επιβλέποντα μήπως χρειάζεται να υπάρχει ξεχωριστό κεφάλαιο με τίτλο ` Σχεδίαση συστήματος '. 

\section{Οργάνωση πειραμάτων}

Εδώ περιγράφουμε λεπτομερώς πώς οργανώσαμε τα πειράματα. Π.χ.
α) τί σύνολο δεδομένων χρησιμοποιήσαμε (συνθετικά, έτοιμες συλλογές)
β) τί τιμές είχαν διάφοροι παράμετροι του συστήματός αξιολόγησης, κ.λ.π.

Οι τιμές των παραμέτρων μπορούν να φαίνονται και σε πίνακα, όπως λ.χ. στον Πίνακα \ref{tab:parameters}:

\begin{table}[h]
\centering
\begin{tabular}{|c|>{\centering\arraybackslash}m{8cm}|}
\hline Πλήθος κελιών καννάβου \textit{\tl{c}} $\times$ \textit{\tl{c}} & 50 $\times$ 50, 100 $\times$ 100, 200 $\times$ 200, \textbf{250} $\times$ \textbf{250}, 500 $\times$ 500, 1000 $\times$ 1000  \\
\hline Τυπική απόκλιση $\sigma$ & 25\tl{m}, 50\tl{m}, 75\tl{m}, \textbf{100\tl{m}}, 150\tl{m}, 200\tl{m} \\
\hline Αριθμός εγγύτερων γειτόνων \textit{\tl{k}} & 1, 2, \textbf{3}, 4, 5, 10, 20 \\
\hline Πιθανοτικό κατώφλι $\theta$ & 50$\%$, 60$\%$, 70$\%$, \textbf{75$\%$}, 80$\%$, 90$\%$, 99$\%$ \\
\hline  
\end{tabular}
\caption{Παράμετροι πειραμάτων}
\label{tab:parameters}
\end{table}

Αν χρησιμοποιήσατε συνθετικά δεδομένα, τότε εξηγήστε στην παρακάτω χωριστή υποενότητα τον τρόπο που τα δημιουργήσατε:

\subsection{Παραγωγή συνθετικών δεδομένων}

Τα πειραματικά δεδομένα παρήχθησαν ...


\section{Αποτελέσματα της μελέτης}

Εδώ παρουσιάζουμε τα αποτελέσματα των μετρήσεων με μορφή γραφικών παραστάσεων, όπως ενδεικτικά στο Σχ. \ref{GridGranularity}. Δίνουμε λεπτομερή εξήγηση και σχολιασμό των αποτελεσμάτων, πάντα σε σχέση με το πρόβλημα που οι τεχνικές μας φιλοδοξούν να λύσουν. 
Φροντίστε να ομαδοποιήσετε τα αποτελέσματα (σε χωριστές υποενότητες)ανάλογα με τις παραμέτρους που μετράτε, π.χ. χωριστά το κόστος σε χρόνο από το κόστος σε χώρο ή όσον αφορά την ακρίβεια των απαντήσεων.

\begin{figure}[t!]
\includegraphics[scale=0.5]{figures/grid_granularity.eps}
\centering
\caption{Κλιμάκωση χρόνου εκτέλεσης για διάφορες υποδιαιρέσεις του καννάβου}	
\label{GridGranularity}
\end{figure} 


\section{Σύνοψη συμπερασμάτων αξιολόγησης}

Εδώ συνοψίζουμε τα συμπεράσματα της αξιολόγησης. Η σύνοψη να γίνεται σύντομα και καθαρά, π.χ. 1. αυτό, 2. το άλλο, κ.ο.κ.


%Αυτά που ακολοθούν θα μπουν στο Κεφ. 5 

\subsubsection{\tl{Server Routes}}

Ο \tl{server} χρησιμοποιεί ένα σύστημα δρομολόγησης (\tl{routes}) των αιτημάτων του \tl{client} που προορίζονται προς εξυπηρέτηση. Κάθε ομάδα υπηρεσιών που απευθύνονται σε μια συγκεκριμένη λειτουργία της εφαρμογής αντιστοιχίζεται σε ένα δρομολογητή (\tl{router}). Ο \tl{router} απαρτίζεται από επιμέρους \tl{routes}, καθεμία εκ των οποίων εξυπηρετεί μια συγκεκριμένη λειτουργία. Έτσι, στο παραπάνω παράδειγμα, το αίτημα για τη δημιουργία νέου \tl{hotspot} θα δρομολογηθεί για εξυπηρέτηση από τον \tl{router} που είναι υπεύθυνος για τις ενέργειες΄που αφορούν ένα \tl{hotspot} (\tl{\textit{HotspotRoutes}}). 

Το σύστημα των δρομολογητών έχει σχεδιαστεί με στόχο την απλότητα και την αμεσότητα στην κατανόηση από τον προγραμματιστή (βλ. Σχ. \ref{routes}). Αιτήματα που αφορούν τους χρήστες δρομολογούνται προς εξυπηρέτηση από τον \tl{router} που είναι υπεύθυνος για τις ενέργειες που αφορούν τους χρήστες. Με τον ίδιο τρόπο, Αιτήματα που σχετίζονται με σχόλια χρηστών σε ένα \tl{hotspot}, εξυπηρετούνται από εκείνο τον \tl{router} που είναι υπεύθυνος για τα σχόλια. 

\begin{figure}[h]
    \centering
    \includegraphics[scale=1]{figures/routes.png}
    \caption{Ιεράρχηση του συστήματος δρομολογητών στην εφαρμογή.}
    \label{routes}
\end{figure}

Οι δρομολογητές του συστήματος είναι οι ακόλουθοι:

\begin{itemize}
    \item \tl{\textbf{Hotspot Routes --}} περιλαμβάνει \tl{routes} που αφορούν την εξυπηρέτηση αιτημάτων σχετικά με \tl{hotspots}, όπως δημιουργία, επεξεργασία, διαγραφή, προβολή, ανάκτηση κλπ.
    \item \tl{\textbf{User Routes --}} περιλαμβάνει \tl{routes} που αφορούν την εξυπηρέτηση αιτημάτων σχετικά με χρήστες, όπως εγγραφή, σύνδεση, επεξεργασία, προβολή, ανάκτηση κλπ.
    \item \tl{\textbf{Comment Routes --}} περιλαμβάνει \tl{routes} που αφορούν την εξυπηρέτηση αιτημάτων σχετικά με σχόλια, όπως δημιουργία, προβολή, απαρίθμηση κλπ.
    \item \tl{\textbf{File Routes --}} περιλαμβάνει \tl{routes} που αφορούν την εξυπηρέτηση αιτημάτων σχετικά με συνημμένα αρχεία σε \tl{hotspots}.
    \item \tl{\textbf{Stats Routes --}} περιλαμβάνει \tl{routes} που αφορούν την εξυπηρέτηση αιτημάτων σχετικά με στατιστικά δεδομένα.
\end{itemize}

\subsubsection{\tl{Server Controllers}}

Αφότου τα αιτήματα δρομολογηθούν προς εξυπηρέτηση στους αντίστοιχους \tl{routers}, ακολοθεί η επεξεργασία τους και η εκτέλεσή τους μέσω κατάλληλων μεθόδων. Οι μέθοδοι που αναλαμβάνουν την εκτέλεση των αιτημάτων ονομάζονται ελεγκτές (\tl{controllers}) και αποτελούνται από ένα σύνολο ασύγχρονων εντολών και κλήσεων μεταξύ της βάσης δεδομένων και του \tl{server}.

Όπως και με το σύστημα των δρομολογητών, έτσι και οι ελεγκτές έχουν σχεδιαστεί με στόχο να αποσυμπλέκουν τις ενέργειες που σχετίζονται με κάθε αίτημα (βλ. Σχ. \ref{}). Όλες οι ενέργεεις που αφορούν τα \tl{hotspots} εκτελούνται αποκλειστικά από έναν ελεγκτή. Το ίδιο συμβαίνει και με τις ενέργεεις γύρω από τους χρήστες. Το σύστημα των \tl{controllers} έχει παραπλήσια μορφή με αυτό των \tl{routers} καθώς στόχος της σχεδίασης είναι να υπάρχει ένας βαθμός αναλογίας μεταξύ των διαφόρων τμημάτων,ο οποίος μειώνει αισθητά το χρόνο κατανόησης και βελτιστοποιεί τη διαδικασία υλοποίησης:

\begin{itemize}
    \item \tl{\textbf{Hotspot Controller --}} περιλαμβάνει τις μεθόδους που αφορούν την εξυπηρέτηση αιτημάτων σχετικά με \tl{hotspots}, όπως δημιουργία, επεξεργασία, διαγραφή, προβολή, ανάκτηση κλπ.
    \item \tl{\textbf{User Controller --}} περιλαμβάνει τις μεθόδους που αφορούν την εξυπηρέτηση αιτημάτων σχετικά με χρήστες, όπως εγγραφή, σύνδεση, επεξεργασία, προβολή, ανάκτηση κλπ.
    \item \tl{\textbf{Comment Controller --}} περιλαμβάνει τις μεθόδους που αφορούν την εξυπηρέτηση αιτημάτων σχετικά με σχόλια, όπως δημιουργία, προβολή, απαρίθμηση κλπ.
    \item \tl{\textbf{File Controller --}} περιλαμβάνει τις μεθόδους που αφορούν την εξυπηρέτηση αιτημάτων σχετικά με συνημμένα αρχεία σε \tl{hotspots}.
    \item \tl{\textbf{Stats Controller --}} περιλαμβάνει τις μεθόδους που αφορούν την εξυπηρέτηση αιτημάτων σχετικά με στατιστικά δεδομένα.
    \item \tl{\textbf{Views Controller --}} περιλαμβάνει τις μεθόδους που αφορούν την εξυπηρέτηση αιτημάτων σχετικά με τα \tl{views} ενός \tl{hotspot}.
\end{itemize}


\include{chapter6}
\chapter{Επίλογος}
\label{chap7}

% Εδώ εξηγούμε ότι θα συνοψίσουμε την μελέτη που εκπονήθηκε στα πλαίσια της διπλωματικής.
Στο παρόν κεφάλαιο συνοψίζεται η συνολική προσπάθεια για την σχεδίαση και την υλοποίηση του συστήματος της εφαρμογής μηνυμάτων σε διεπαφή χάρτη με τη βοήθεια του πληθοπορισμού. Θα εκφραστούν οι σκέψεις για μελλοντικές επεκτάσεις της εφαρμογής, για περαιτέρω λειτουργικότητες που μπορούν να προστεθούν, καθώς επίσης και για ανάγκες που μπορούν να εξυπηρετηθούν.

\section{Τελικό Σύστημα}
Το σύστημα σχεδιάστηκε με τέτοιο τρόπο ώστε να είναι ευέλικτο και γρήγορο στη χρήση. Από την πρώτη επαφή με την εφαρμογή, ο χρήστης εισάγεται στο νόημα της εφαρμογής άμεσα και αναλαμβάνει το ρόλο που του αναθέτει η εφαρμογή αβίαστα και χωρίς δυσκολία. Σε αυτό συνέβαλλαν τόσο οι σχεδιαστικές τεχνικές που χρησιμοποιήθηκαν, όσο και οι σύγχρονες τεχνολογίες με τις οποίες υλοποιήθηκαν οι διάφορες διεπιφάνειες της εφαρμογής. Η εξέλιξη πλέον των αρχιτεκτονικών σχεδίασης, του διαδικτύου, αλλά και των δυνατοτήτων των καθημερινών υπολογιστών είναι τέτοιες που η ανάπτυξη εφαρμογών δεν απαιτεί πολλά έξοδα, ενώ οι εφαρμογές δεν στερούνται τίποτα σε ταχύτητα ή πολυπλοκότητα. Επιπρόσθετα, η εκμάθηση των \tl{frameworks} είναι πλέον σχετικά σύντομη και εύκολη, ενώ οι δυνατότητες που παρέχουν είναι πολύ εξελιγμένες και έτσι πολλές εφαρμογές μπορούν να παραχθούν σχετικά εύκολα και αποδοτικά. Μέσω αυτών των δυνατοτήτων το υλοποιημένο σύστημα μπόρεσε να παράσχει στους χρήστες της εφαρμογής τη δυνατότητα να δημιουργούν δημοσιεύσεις και σχόλια, ενώ δόθηκε και ένα ολοκληρωμένο κοινωνικό δίκτυο με σκοπό την επικοινωνία και παρακολούθηση των δράσεων μεταξύ χρηστών και εφαρμογής. Η δωρεάν διάθεση της εφαρμογής σε όλους τους χρήστες, καθιστά προσβάσιμη την απόκτησή της και τη χρήση της σε οποιαδήποτε τοποθεσία διαθέτει κάποιο τρόπο σύνδεσης στο διαδίκτυο. Η απλή επίσης σχεδιαστική τεχνοτροπία βοηθάει τους άπειρους χρήστες να μάθουν γρήγορα το σύστημα, χωρίς έτσι να απαιτείται μεγάλος χρόνος εκμάθησης.


\section{Μελλοντικές επεκτάσεις}

Το παραπάνω σύστημα μπορεί να χρησιμοποιηθεί ως μια βάση ώστε να
εμπλουτιστεί με νέες δυνατότητες οι οποίες να παρέχουν στο χρήστη μια πιο ολοκληρωμένη εμπειρία σχετικά με την ενημέρωση σε θέματα πολιτισμικού περιεχομένου. Μερικά μελλοντικά θέματα προς διερεύνηση θα μπορούσαν να είναι:

\begin{itemize}
    \item Δυνατότητα επισύναψης \tl{3D} αντικειμένων στις δημοσιεύσεις του χρήστη, τα οποία θα μπορούν να είναι φωτογραφίες,αρχεία βίντεο ή άλλης υποστηριζόμενης μορφής αρχεία (πχ. \tl{.fbx, .obj, .3ds} κλπ.).
    \item Δυνατότητα αλληλεπίδρασης του χρήστη με τα \tl{3D} αρχεία μιας δημοσίευσης, όπως προβολή, σχολιασμός και απάντηση με νέο μήνυμα αρχείου.
    \item Επιλογή αλληλεπίδρασης με την προσθήκη ενός \tl{like button}.
    \item Εμφάνιση χρονικής διάρκειας ισχύος στις λεπτομέρειες κάθε \tl{hotspot} και διαγραφή μετά τη λήξη αυτού, ο χάρτης ανανεώνεται διαρκώς με τα ισχύοντα \tl{hotspots}.
    \item Προσθήκη χρηστών σε λίστα φίλων για γρήγορη εύρεση και προβολή των δημοσιεύσεών τους.
    \item Δυνατότητα αποστολής μυνημάτων σε φίλους με την βοήθεια ενσωματωμένης πλατφόρμας γι' αυτό το σκοπό.
\end{itemize}




%OPTION #1: Embed bibliography from file `references.tex' using plain references.

\begin{thebibliography}{1}

\addcontentsline{toc}{chapter}{Βιβλιογραφία}

\bibitem{[AMP+14]} {\textlatin{
{Jeff Sondermann, American Press Institute}.
Mobile and social media are intricately linked.
\url{https://www.americanpressinstitute.org/publications/reports/white-papers/mobile-and-social-media/ }
  2014. Last accessed on 07/03/2019}}.
  
\bibitem{[VEN+18]} {\textlatin{
{Ricardo Bilton}.
New data shows just how much social sharing has decreased since 2015 (and News Feed tweaks are just one factor).
\url{https://www.venturelean.team/hello-world/ }
  2018. Last accessed on 08/03/2019}}.
  
\bibitem{[BBC+18]} {\textlatin{
{Jessica Brown, BBC Future}.
Is social media bad for you? The evidence and the unknowns.
\url{http://www.bbc.com/future/story/20180104-is-social-media-bad-for-you-the-evidence-and-the-unknowns }
  2018. Last accessed on 08/03/2019}}.

\bibitem{[CSW+18]} {\textlatin{
{CrowdSourcingWeek}.
What is Crowdsourcing?
\url{https://crowdsourcingweek.com/what-is-crowdsourcing/ }
  2018. Last accessed on 08/03/2019}}.
  
\bibitem{[WIT+18]} {\textlatin{
{ISLAB Team, NTUA}.
About WITH
\url{http://withcrowd.eu/about }
  2018. Last accessed on 08/03/2019}}.
 
\bibitem{[4SQ+18]} {\textlatin{
{Foursquare Developer Team}.
Foursquare API
\url{https://developer.foursquare.com/docs/api/endpoints }
  2018. Last accessed on 08/03/2019}}.
  
\bibitem{[IND+16]} {\textlatin{
{Grace Fearon, iStudent}.
Our need to maintain social approval is actually making us lose what is best about ourselves - our individuality.
\url{https://www.independent.co.uk/student/istudents/our-need-to-maintain-social-approval-is-actually-making-us-lose-what-is-best-about-ourselves-our-a6827316.html }
   2016}}.
  
\bibitem{[JAR+18]} {\textlatin{
{Mansi Beniwal}.
Social Media and its Impact in Interpersonal Relationships.
\url{https://jarvee.com/social-media-impact-interpersonal-relationships/ }
  2018}}.
  
\bibitem{[SQA+07]} {\textlatin{
{Ullah, Asmat}.
Client Side Scripting for Web Applications
\url{https://www.sqa.org.uk/e-learning/SiteHomeCD/page_26.htm }
  2007}}.
  
\bibitem{[STR+09]} {\textlatin{
{Stroustrup, Bjarne}.
Bjarne Stroustrup's FAQ: What do you think of C++/CLI?
\url{http://www.stroustrup.com/bs_faq.html#CppCLI}
  2009}}.
  
\bibitem{[DEV+03]} {\textlatin{
{Gregory, Kate}.
Managed, Unmanaged, Native: What Kind of Code Is This?
\url{https://www.developer.com/net/cplus/article.php/2197621 }
  2003}}.
  
\bibitem{[MIC05]} {\textlatin{
Meijer, Erik and Peter Drayton. Static Typing Where Possible, Dynamic Typing When Needed: The End of the Cold War Between Programming Languages. Microsoft Corporation, 2005}}.  

\bibitem{[ADV09]} {\textlatin{
L.~Tratt. Dynamically Typed Languages. {\em Advances in Computers}, 77(2):149--184, July 2009}}.  
 
\bibitem{[SWIFT1+16]} {\textlatin{
{Swift.org, Apple Inc.}.
The Swift Linux Port
\url{https://swift.org/blog/swift-linux-port/}
  2016}}.
  
\bibitem{[SWIFT2+14]} {\textlatin{
{Chris Lattner}.
Chris Lattner's Homepage
\url{http://nondot.org/sabre/}
  2014}}.
  
\bibitem{[SWIFT3+14]} {\textlatin{
{Apple Worldwide Developers Conference, Session 102}.
Platforms State of the Union
  2016}}.

\bibitem{[SWIFT4]} {\textlatin{
{ Rachel Metz, MIT Technology}.
Apple Seeks a Swift Way to Lure More Developers
\url{https://www.technologyreview.com/s/527821/apple-seeks-a-swift-way-to-lure-more-developers/}
  2014}}.
  
\bibitem{[SWIFT5]} {\textlatin{
{Harrison Weber, VentureBeat}.
Apple announces 'Swift', a new programming language for macOS & iOS
\url{https://venturebeat.com/2014/06/02/apple-introduces-a-new-programming-language-swift-objective-c-without-the-c/}
  2014}}.

\bibitem{[JAVA1]} {\textlatin{
{J.~Gosling, J.~Bill, G.~Steele, G.~Bracha, A.~Buckley}.
The Java® Language Specification
{\em Java SE 8th edition}, 2014}}.

\bibitem{[JAVA2]} {\textlatin{
{Computer Weekly}.
Write once, run anywhere?
\url{http://www.computerweekly.com/Articles/2002/05/02/186793/write-once-run-anywhere.htm}
  2002}}.
  
\bibitem{[JAVA3]} {\textlatin{
{Oracle Inc.}.
Design Goals of the Java™ Programming Language
\url{https://www.oracle.com/technetwork/java/intro-141325.html}
  2013}}.
  
\bibitem{[JAVA4]} {\textlatin{
{R.~McMillan, wired.com}.
Is Java Losing Its Mojo?
\url{https://www.wired.com/2013/01/java-no-longer-a-favorite/}
  2013}}.
  
\bibitem{[JAVA5]} {\textlatin{
{Stephen O'Grady, RedMonk}.
The RedMonk Programming Language Rankings: January 2015
\url{https://redmonk.com/sogrady/2015/01/14/language-rankings-1-15/}
  2015}}.
  
\bibitem{[JAVA6]} {\textlatin{
{langpop.com}.
Programming Language Popularity
\url{https://web.archive.org/web/20090116080326/http://www.langpop.com/}
  2009}}.
  
\bibitem{[JAVA7]} {\textlatin{
{Tata McGraw}. 
Object-oriented Programming with Java: Essentials and Applications.
{\em Hill Education}, 1:30--35, 2009}}.  

\bibitem{[JAVA8]} {\textlatin{
{Sun Microsystems, WaybackMachine}.
JAVASOFT SHIPS JAVA 1.0 - {\em
Programming environment available free for developers}
\url{https://web.archive.org/web/20070310235103/http://www.sun.com/smi/Press/sunflash/1996-01/sunflash.960123.10561.xml}
  2018}}.  
  
\bibitem{[JAVA9]} {\textlatin{
{Alex Mullis, Android Authority}.
How to install the Android SDK (Software Development Kit)
\url{https://www.androidauthority.com/how-to-install-android-sdk-software-development-kit-21137/}
  2016}}. 
  
\bibitem{[JAVA10]} {\textlatin{
{Android Developers}.
Introduction to Android
\url{https://developer.android.com/guide/index.html}
  2017}}.  
\bibitem{[JAVA11]} {\textlatin{
{Android Developers}.
Tools Overview
\url{https://developer.android.com/studio/command-line/}
  2012}}.  





\bibitem{[ACC+03]} {\textlatin{
D.~J. Abadi, D.~Carney, U.~{\c{C}}etintemel, M.~Cherniack, C.~Convey, S.~Lee,  M.~Stonebraker, N.~Tatbul, and S.~Zdonik. 
Aurora: a new model and architecture for data stream management.
{\em The VLDB Journal -- The International Journal on Very Large Data Bases}, 12(2):120--139, 2003}}.

\bibitem{[DRS09]} {\textlatin{
N.~Dalvi, C.~R{\'e}, and D.~Suciu. Probabilistic databases: Diamonds in the dirt. {\em Communications of the ACM}, 52(7):86--94, July 2009}}.

\bibitem{[GBE+00]} {\textlatin{
R.~H. G{\"u}ting, M.~H. B{\"o}hlen, M.~Erwig, C.~S. Jensen, N.~A. Lorentzos,
  M.~Schneider, and M.~Vazirgiannis.
A foundation for representing and querying moving objects.
{\em ACM Transactions on Database Systems (TODS)}, 25(1):1--42, 2000}}.

\bibitem{[JMS+08]} {\textlatin{
N.~Jain, S.~Mishra, A.~Srinivasan, J.~Gehrke, J.~Widom, H.~Balakrishnan,
  U.~{\c{C}}etintemel, M.~Cherniack, R.~Tibbetts, and S.~Zdonik.
Towards a streaming {SQL} standard.
In {\em Proceedings of the VLDB Endowment}, 1(2):1379--1390, 2008}}.

\bibitem{[MHP05]} {\textlatin{
K.~Mouratidis, D.~Papadias, and M.~Hadjieleftheriou. 
Conceptual partitioning: an efficient method for continuous nearest
  neighbor monitoring.
In {\em Proceedings of the 24th ACM SIGMOD International
  Conference on Management of Data}, pages 634--645, 2005}}.

\bibitem{[Ora11]} {\textlatin{
{Oracle, Inc}.
Complex event processing {CQL} language reference.
\url{http://docs.oracle.com/cd/E16764_01/doc.1111/e12048/intro.htm }
  2009. Last accessed on 15/09/2013}}.
  

\bibitem{[PS11]} {\textlatin{
K.~Patroumpas and T.~Sellis.
Subsuming multiple sliding windows for shared stream computation.
In {\em Advances in
  Databases and Information Systems}, volume 6909 of Springer {\em Lecture Notes in
  Computer Science}, pages 56--69, 2011}}.

\bibitem{[RSV02]} {\textlatin{
P.~Rigaux, M.~Scholl, and A.~Voisard.
{\em Spatial databases: with application to {GIS}}.
Morgan Kaufmann, 2001}}.

\bibitem{[Pap15]}
Σεραφείμ Παπαδιάς.
Απόρριψη φόρτου από ρεύματα
  τροχιάς κινούμενων αντικειμένων. Διπλωματική Εργασία {\textlatin{\em DIPL-2015-02}} στο
Εργαστήριο Συστημάτων Βάσεων Γνώσεων και Δεδομένων, Εθνικό Μετσόβιο
  Πολυτεχνείο, Ιούλιος 2015.

\end{thebibliography}
%\addcontentsline{toc}{chapter}{Βιβλιογραφία}

%OPTION #2: Alternatively, prepare properly formatted BibTeX entries in file `references.bib'. 
%After processing with BibTeX, a file `main.bbl' is automatically populated and it is actually used for producing references in the resulting pdf. 
%IMPORTANT: You must manually modify `main.bbl' by adding \selectlanguage{english} (TOP) and \selectlanguage{english} (BOTTOM) in order to correctly display Latin and Greek characters in the final text.
%\bibliography{references}


%\appendix
%\include{proofs}
\selectlanguage{english}

\chapter*{\selectlanguage{greek}Κατάλογος Ακρονύμων}
\label{abbreviations}

\addcontentsline{toc}{chapter}{\selectlanguage{greek}Κατάλογος Ακρονύμων}
\begin{acronym}[WWWDC] % Give the longest label here so that the list is nicely aligned

\acro{IT}{Information Technology}
\acro{NGO}{Non Governmental Organization}
\acro{API}{Application programming Interface}
\acro{URL}{Uniform Resource Locator}
\acro{SDK}{Software Development Kit}
\acro{CLR}{Common Language Runtime}
\acro{CIL}{Common Intermediate Language}
\acro{OS}{Operating System}
\acro{IDE}{Integrated Development Environment}
\acro{WWDC}{Worldwide Development Conference}
\acro{OO}{Object Oriented}
\acro{WORA}{Write Once Run Anywhere}
\acro{JS}{JavaScript}
\acro{ES}{EcmaScript}
\acro{HTML}{Hypertext Markup Language}
\acro{CSS}{Cascading Style Sheets}
\acro{DOM}{Document Object Model}
\acro{AJAX}{Asynchronous Javascript And XML}
\acro{I/O}{Input Output}
\acro{SPA}{Single Page Application}
\acro{MVC}{Model View Controller}
\acro{APP}{Application}
\acro{PROP}{Property}
\acro{JSX}{JavaScript and XML}
\acro{VDOM}{Virtual Document Object Model}
\acro{UI}{User Interface}
\acro{XML}{Extensible Markup Language}
\acro{JSON}{JavScript Object Notation}
\acro{PHP}{Hypertext preprocessor}
\acro{NPM}{Node Package Manager}
\acro{RAM}{Random Access Memory}
\acro{GB}{GigaByte}
\acro{MB}{MegaByte}
\acro{1M}{One Million}
\acro{REST}{Representational State Transfer}
\acro{HTTP}{Hypertext Tranfer Protocol}
\acro{FS}{File System}
\acro{OAuth}{Open Authentication}
\acro{JWT}{JSON Web Token}
\acro{FAB}{Floating Action Button}
\acro{OO}{Object Oriented}
\acro{OO}{Object Oriented}
\acro{OO}{Object Oriented}
\acro{OO}{Object Oriented}
\acro{OO}{Object Oriented}
\acro{OO}{Object Oriented}
\end{acronym}

\selectlanguage{greek}
\newcommand{\gloss}[2]{\en{#2} \> #1\\ }

\chapter*{\selectlanguage{greek}Γλωσσάριο}
\label{glossary}

\addcontentsline{toc}{chapter}{\selectlanguage{greek}Γλωσσάριο}

\begin{tabbing}
%τα 'a' ρυθμίζουν το πλάτος των δύο στηλών
  aaaaaaaaaaaaaaaaaaaaaaaaaaaaaaaa \= aaaa\kill
  
  \Large\textbf{Αγγλικός όρος} \> \Large\textbf{Ελληνικός όρος} \\
  \gloss{μέσα κοινωνικής δικτύωσης}{social media}
  \gloss{μία από τις δημοφιλέστερες εταιρείες και υπηρεσίες}{facebook}
  \gloss{κοινωνικής δικτύωσης σήμερα}{}
  \gloss{υπηρεσία κοινωνικής δικτύωσης για κοινή χρήση}{instagram}
  \gloss{φωτογραφιών και βίντεο}{}
  \gloss{εφαρμογή ανταλλαγής μηνυμάτων πολυμέσων}{snapchat}
  \gloss{αμερικανική ηλεκτρονική υπηρεσία ειδήσεων}{twitter}
  \gloss{και κοινωνικής δικτύωσης}{}
  \gloss{εφαρμογή για κοινωνική αναζήτηση βασισμένη}{tinder}
  \gloss{σε τοποθεσίες}{}
  \gloss{αμερικανική εφημερίδα}{New York Times}
  \gloss{πληθοπορισμός}{crowdsourcing}
  \gloss{ψηφιακή πλατφόρμα αρχειοθέτησης ευρυμάτων}{withcrowd}
  \gloss{της ευρωπαϊκής πολιτισμικής κληρονομιάς}{}
  \gloss{πύλη}{portal}
  \gloss{ψηφιακή αποθήκη}{repository}
  \gloss{εφαρμογή για κοινή χρήση βίντεο}{youtube}
  \gloss{εφαρμογή τοπικής αναζήτησης κοινωνικών τόπων}{foursquare}
  \gloss{τελικό σημείο}{endpoint}
  \gloss{εφαρμογή που τρέχει σε φυλλομετρητή}{web application}
  \gloss{μητρική εφαρμογή}{native application}
  \gloss{γλώσσα προγραμματισμού}{programming language}
  \gloss{διαχειριζόμενη}{managed}
  \gloss{μεταγλωττισμένη}{compiled}
  \gloss{δυναμική}{dynamic}
  \gloss{μητρική}{native}
  \gloss{μεταγλωττιστής}{compiler}
  \gloss{υψηλού επιπέδου}{high-level}
  \gloss{χρόνος καθυστέρησης}{overhead}
  \gloss{χαμηλού επιπέδου}{low-level}
  \gloss{μεταξύ πολλών πλατφόρμων}{cross-platform}
  \gloss{κώδικας ψηφίων}{bytecode}
  \gloss{λειτουργικό σύστημα που ανήκει στην \selectlanguage{english}Apple Inc.\selectlanguage{greek}}{ios}
  \gloss{λειτουργικό σύστημα που ανήκει στην \selectlanguage{english}Google Inc.\selectlanguage{greek}}{android}
  \gloss{προγραμματιστής}{developer}
  \gloss{ταυτόχρονος}{concurrent}
  \gloss{κλάσση}{class}
  \gloss{φυλλομετρητής ιστού}{web browser}
  \gloss{ιστοσελίδα}{web page}
  \gloss{αντικείμενο}{object}
  \gloss{ακέραιος αριθμός}{integer}
  \gloss{μεταβλητή κινητής υποδιαστολής}{floating point}
  \gloss{μεταβλητή λογικής τιμής}{boolean value}
  \gloss{έξυπνη κινητή συσκευή}{smartphone}
  \gloss{φορητή συσκευή}{tablet}
  \gloss{διορθωτής σφαλμάτων}{debugger}
  \gloss{προσομοιωτής}{emulator}
  \gloss{φορητή συσκευή}{tablet}
  \gloss{συναρτησιακός}{functional}
  \gloss{οδηγούμενο από γεγονότα}{event-driven}
  \gloss{πλευρά πελάτη}{client-side}
  \gloss{αντικειμενοστραφής}{object-oriented}
  \gloss{ξεθώριασμα}{fade out/in}
  \gloss{ολίσθηση}{slide out/in}
  \gloss{αντικείμενα κώδικα}{script-objects}
  \gloss{αντικείμενα φιλοξενίας}{host-objects}
  \gloss{εξυπηρετητής/διακομιστής}{server}
  \gloss{αιτήματα}{requests}
  \gloss{βρόχος συμβάντων}{event loop}
  \gloss{εκτέλεση χωρίς μπλοκάρισμα}{non-blocking execution}
  \gloss{συνάρτηση ανάκλησης}{callback function}
  \gloss{περιεχόμενο ανάκλησης}{callback content}
  \gloss{λογική πλήρους διεκπαιρέωσης}{run to completion logic}
  \gloss{περιηγητής}{browser}
  \gloss{η πλευρά του εξυπηρετητή}{backend}
  \gloss{η πλευρά του πελάτη}{frontend}
  \gloss{εφαρμογή επιφάνειας εργασίας}{desktop application}
  \gloss{μοντέλο}{model}
  \gloss{όψη}{view}
  \gloss{ελεγκτής}{controller}
  \gloss{κλάση-δοχείο}{container class}
  \gloss{λογική παρουσίασης}{presentation logic}
  \gloss{εφαρμογή για φορητή συσκευή}{mobile application}
  \gloss{στοιχείο}{component}
  \gloss{αρχιτεκτονική βασισμένη σε στοιχεία}{component-based architecture}
  \gloss{ιδιότητα}{property}
  \gloss{μονόδρομη ροή δεδομένων}{one-way data binding}
  \gloss{κατάσταση}{state}
  \gloss{στοιχείο με κατάσταση}{stateful component}
  \gloss{στοιχείο-παιδί}{child-component}
  \gloss{στοιχείο-πατέρας}{parent-component}
  \gloss{συναρτήσεις-γάντζοι}{hook}
  \gloss{συνάρτηση που εκτελείται σε κάποιο συγκεκριμένο}{lifecycle method}
  \gloss{χρονικό σημείο κατά τον κύκλο ζωής της εφαρμογής}{}
  \gloss{πλαίσιο}{framework}
  \gloss{προγραμματιστική εμπειρία}{developer experience}
  \gloss{επαναμεταγλώττιση}{rebuild}
  \gloss{σχεδίαση}{design}
  \gloss{περιηγητής της \selectlanguage{english}Google\selectlanguage{greek}}{Chrome}
  \gloss{περιηγητής της \selectlanguage{english}Apple\selectlanguage{greek}}{Safari}
  \gloss{επεξεργαστής κειμένου}{text editor}
  \gloss{επαναχρησιμότητα κώδικα}{code-reuse}
  \gloss{κοινή χρήση γνώσης}{knowledge sharing}
  \gloss{εξοικονόμηση πόρων}{resource saving}
  \gloss{ιδιωτική ηλεκτρονική εταιρεία που πρεοσφέρει}{airbnb}
  \gloss{υπηρεσίες διαμονής και φιλοξενίας}{}
  \gloss{εφαρμογή τηλεπικοινωνίας}{skype}
  \gloss{γέφυρα}{bridge}
  \gloss{βασίλειο, σφαίρα, εδώ: κατηγορία}{realm}
  \gloss{διαλειτουργική γλώσσα}{interoperable language}
  \gloss{κόμβος}{node}
  \gloss{ουρά συμβάντων}{event queue}
  \gloss{νήμα-εργάτης}{worker-thread}
  \gloss{δεξαμενή νημάτων εργασίας}{worker-thread pool}
  \gloss{μεταγλώττιση}{interpreting}
  \gloss{παραστατική μεταβίβαση κατάστασης}{repersentational state transfer}
  \gloss{δρομολογητής}{router}
  \gloss{διαδρομή}{route}
  \gloss{φορτιο απόκρισης}{response payload}
  \gloss{κλειδί}{key}
  \gloss{τιμή}{value}
  \gloss{συμβολοσειρά}{string}
  \gloss{λογική τιμή}{boolean}
  \gloss{πίανακας}{array}
  \gloss{κενός χαρακτήρας}{null}
  \gloss{μη-σχετικιστικός}{non-relational}
  \gloss{εγγραφοστραφής}{document-oriented}
  \gloss{μοντέλο εγγράφου σε \tl{MongoDB}}{schema}
  \gloss{συλλογή}{collection}
  \gloss{αναζήτηση συγκεκριμένου σκοπού}{ad hoc query}
  \gloss{αναζήτηση με βάση πεδίο}{field search}
  \gloss{αναζήτηση με βάση συνθήκη}{range search}
  \gloss{αναζήτηση με βάση μοτίβο}{regular expression search}
  \gloss{αναπαραγωγή/αντιγραφή}{replication}
  \gloss{σύνολο, ομάδα}{set}
  \gloss{δεικτοδότηση}{indexing}
  \gloss{εξισορρόπηση φορτίου}{load balancing}
  \gloss{κοπή, τεμαχιοποίηση}{sharding}
  \gloss{διαδικασία κρυπτογράφησης δεδομένων}{hashing}
  \gloss{αποθήκευση αρχείων}{file storage}
  \gloss{άθροιση}{aggregation}
  \gloss{αγωγός}{pipeline}
  \gloss{μέθοδος αθροιστικής χαρτογράφησης}{map-reduce function}
  \gloss{άθροιση μονού σκοπού}{single-purpose aggregation}
  \gloss{σελιδοποίηση}{pagination}
  \gloss{σύστημα ταυτοποίησης ανοικτού τύπου}{open authentication}
  \gloss{σύμβολο πιστοποίησης}{access token}
  \gloss{όνομα χρήστη}{username}
  \gloss{ηλεκτρονική διεύθυνση}{email}
  \gloss{κωδικός πρόσβασης}{password}
  \gloss{σύμβολο ανανέωσης}{refresh token}
  \gloss{φόρμα εγγραφής}{register/signup form}
  \gloss{χειραψία, διαδικασία ανταλλαγής}{handshake}
  \gloss{αποσύνδεση}{logout}
  \gloss{φόρμα σύνδεσης}{login/sigin form}
  \gloss{επικεφαλίδα εξουσιοδότησης}{authorization header}
  \gloss{στρατηγική ταυτοποίησης χρηστών}{passport strategy}
  \gloss{βασική δομή κώδικα}{boilerplate}
  \gloss{προφίλ χρήστη}{profile}
  \gloss{σημείο ενδιαφέροντος στην εφαρμογή}{hotspot}
  \gloss{σχόλιο}{comment}
  \gloss{σημείο ενδιαφέροντος}{point of interest}
  \gloss{απόδοση}{performance}
  \gloss{ανάκτηση}{recoverability}
  \gloss{βιβλιογραφία, υλικό υποστήριξης}{documentation}
  \gloss{ασφάλεια}{security}
  \gloss{επεκτασιμότητα}{scalability}
  \gloss{η διαδικασία κατά την οποία μια εφαρμογή}{hosting}
  \gloss{φιλοξενείται από έναν κεντρικό \tl{server}}{}
  \gloss{σενάρια χρήσης}{use cases}
  \gloss{οθόνη}{screen}
  \gloss{εγγραφή}{register}
  \gloss{σύνδεση}{login}
  \gloss{πλήρες όνομα}{fullname}
  \gloss{επιβεβαίωση κωδικού πρόσβασης}{confirm password}
  \gloss{ημερομηνία γέννησης}{birthdate}
  \gloss{πόλη}{city}
  \gloss{γένος}{gender}
  \gloss{περιγραφή}{description}
  \gloss{χρονική διάρκεια ισχύος}{validity}
  \gloss{καφές}{coffee}
  \gloss{φαγητό}{food}
  \gloss{ποτά}{drinks}
  \gloss{αξιοθέατα}{sights}
  \gloss{τέχνες}{arts}
  \gloss{επαναφορά στην τοποθεσία χρήστη}{find my location}
  \gloss{στατιστικά δεδομλένα λογαριασμού}{account stats}
  \gloss{πρόσθετες πληροφορίες}{insights}
  \gloss{κοινό}{audience}
  \gloss{αγωγός}{pipeline}
  \gloss{αγωγός}{pipeline}
  \gloss{αγωγός}{pipeline}
  \gloss{αγωγός}{pipeline}
  \gloss{αγωγός}{pipeline}
  \gloss{αγωγός}{pipeline}
  \gloss{αγωγός}{pipeline}
  \gloss{αγωγός}{pipeline}
  \gloss{αγωγός}{pipeline}
  \gloss{αγωγός}{pipeline}
  \gloss{αγωγός}{pipeline}
  \gloss{αγωγός}{pipeline}
  \gloss{αγωγός}{pipeline}
  \gloss{αγωγός}{pipeline}
  \gloss{αγωγός}{pipeline}
  \gloss{αγωγός}{pipeline}
\end{tabbing}


\backmatter
\printindex

\end{document}
